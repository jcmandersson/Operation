\section{Jonas Andersson}
\subsection{Inledning} 
Tidigare när jag har utvecklat webbprogram så har jag använt mig av PHP tillsammans med HTML, css och javascript. Jag har dessutom valt att skriva mycket själv och inte förlita mig på ramverk och bibliotek. I detta projekt valde jag, som arkitekt, istället att byta ut PHP mot node.js. Dessutom har jag lagt mycket vikt på använda bibliotek så mycket som möjligt för att slippa uppfinna hjulet på nytt.
\subsubsection{Syfte}
Syftet med denna del av rapporten är att analysera vad det finns för fördelar och nackdelar med att använda node.js gentemot andra vanliga språk för webben. Det ska även undersökas hur inlärningskurvan beror på språk och externa ramverk och bibliotek.
\subsubsection{Frågeställning}
Frågeställningar
\begin{itemize}
  \item Vad finns det för fördelar/nackdelar med node.js gentemot andra språk för webben?
  \item Vad finns det för fördelar/nackdelar med att använda mycket externa bibliotek?
\end{itemize}
\subsubsection{Avgränsningar}
Det finns många programmeringsspråk som man kan använda i samma syfte som node.js. Eftersom jag enbart har tidigare erfarenheter av PHP och Python så kommer node.js jämföras mot dessa och inga andra. 
\subsection{Bakgrund}
\subsection{Teori}
Node.js är en plattform för att skapa applikationer till framförallt webbservrar. Det finns en inbyggd pakethanterare vid namn npm som gör det enkelt att inkludera både små och stora bibliotek i sina projekt. Det är därför väldigt enkelt att använda sig av ett bibliotek istället för att skriva all funktionalitet själv.\\

Javascript är det programmeringsspråk som används i Node.js. På klienten, d.v.s. i webbläsaren, är man tvingad att använda sig av javascript eller något programmeringsspråk som kan kompileras till javascript. Genom att använda node.js får man därav samma språk på både server och klient. \\
\subsection{Metod}

\subsection{Resultat}
\subsection{Diskussion}
\subsubsection{Resultat}
\subsubsection{Metod}
\subsection{Slutsatser}