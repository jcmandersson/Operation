\section{Jonas Andersson}
\subsection{Inledning} 
Tidigare när jag har utvecklat webbprogram så har jag använt mig av PHP tillsammans med HTML, css och javascript. Jag har dessutom valt att skriva mycket själv och inte förlita mig på ramverk och bibliotek. I detta projekt valde jag, som arkitekt, istället att byta ut PHP mot node.js. Dessutom har jag lagt mycket vikt på använda bibliotek så mycket som möjligt.
\subsubsection{Syfte}
Syftet med denna del av rapporten är att analysera vad det finns för fördelar och nackdelar med att använda node.js gentemot andra vanliga språk för webben. Det ska även undersökas hur inlärningskurvan beror på språk och externa ramverk och bibliotek.
\subsubsection{Frågeställning}
\begin{itemize}
  \item Vad finns det för fördelar/nackdelar med node.js gentemot andra språk för webben?
  \item Kan man förkorta inlärningskurvan till webbprogrammering genom att använda samma språk till både front-end och back-end?
  \item Kan man förkorta inlärningskurvan till webbprogrammering genom att använda mycket ramverk och bibliotek?
\end{itemize}
\subsubsection{Avgränsningar}
Det finns många programmeringsspråk som man kan använda i samma syfte som node.js. Eftersom jag enbart har tidigare erfarenheter av PHP och Python så kommer node.js jämföras mot dessa och inga andra. Av samma anledning begränsas rapporten till att bara undersöka ett ramverk, i detta fall keystonejs.
\subsection{Bakgrund}
\subsection{Teori}
\subsection{Metod}
\subsection{Resultat}
\subsection{Diskussion}
\subsubsection{Resultat}
\subsubsection{Metod}
\subsection{Slutsatser}