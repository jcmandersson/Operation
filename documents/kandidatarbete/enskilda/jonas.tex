\section{Jonas Andersson}
\subsection{Inledning} 

Tidigare när jag har utvecklat webbprogram så har jag använt mig av PHP tillsammans med HTML, CSS och Javascript. Jag har även valt att skriva det mesta själv och inte använt mig av så mycket ramverk och bibliotek. I detta projekt valde jag, som arkitekt, att byta ut PHP mot Node.js och att istället för ren HTML och CSS använda oss av Handlebars och Less. Dessutom har jag lagt mycket vikt på att använda många andra bibliotek och ramverk för att slippa uppfinna hjulet på nytt.

\subsubsection{Syfte}

Syftet med denna del av rapporten är att analysera om vad det finns för fördelar och nackdelar med att använda Node.js till ett sådant här projekt. Det ska även undersökas vad det finns för fördelar och nackdelar med att använda många externa ramverk och bibliotek.

\subsubsection{Frågeställning}
Frågeställningar
\begin{itemize}
	\item Vad finns det för fördelar/nackdelar med att använda Node.js till detta projekt?
	\item Vad finns det för fördelar/nackdelar med att använda så många externa bibliotek som vi gjorde i detta projekt?
\end{itemize}
\subsubsection{Avgränsningar}
Det finns många programmeringsspråk som man kan använda i samma syfte som Node.js. Eftersom jag enbart har tidigare erfarenheter av PHP och Python så kommer alla jämförelser av Node.js vara mot dessa och inga andra. Exempelvis vägs fördelar och nackdelar mot PHP och Python.
\subsection{Bakgrund}

I detta projekt har jag haft rollen som arkitekt och alltså tagit fram grundstenarna till programmet. Jag valde en arkitektur byggd på Node.js med många bibliotek och ramverk. Node.js valdes framförallt av två anledningar. 

Först och främst så kom alla in i projektet med olika erfarenheter och kunskaper inom webbprogrammering och vissa hade aldrig tidigare skapat en hemsida. Trots detta ville vi snabbt komma igång och jag ville därför ta fram en arkitektur som var så lite nytt som möjligt. Eftersom man på ett eller annat sätt är tvungen att skriva Javascript när man programmerar till webben så tyckte jag det var bra att hålla sig till så få nya språk som möjligt.

Den andra anledning till att jag valde Node.js beror på kunden. Visserligen hade inte kunden några specifika krav på arkitekturen men däremot är vissa av kraven i kravspecifikationen olika lätta att implementera beroende på vilken plattform man väljer. Det fanns ett krav som gick ut på att flera personer skulle kunna kryssa av en lista samtidigt och alla som var inne på sidan skulle se uppdateringen i realtid. Detta kan vara svårt att lösa på många plattformar men blir väldigt enkelt med websockets som Node.js har bra stöd för. Detta till skillnad mot exempelvis PHP där websocket är krångligare att använda.

I och med att jag inte hade så mycket erfarenhet av Node.js och att använda många bibliotek sedan tidigare så var det svårt att veta om det skulle passa bra till detta projekt. Dessutom hade jag ingen tidigare erfarenhet av webbprogrammering i grupp. Under projektets gång har jag samlat på mig mer erfarenheter som underlättar för att göra en bedömning av detta arkitekturval.

%Skriv om: 
%-* Olika erfarenheter och kunskaper
%-* Ville komma igång snabbt
%-* Socket.IO - Engångsartiklar
%-* Mina tidigare erfarenheter (av Javascript, ...)
%* Få en bättre röd tråd här...
%-* Varför jag valde arkitekturen

\subsection{Teori}

\subsubsection{PHP}
%http://php.net/manual/en/history.php.php

PHP är ett programmeringsspråk som man kan skriva direkt i sin HTML-kod. Det skapades från början för att göra det enklare att lägga in funktionalitet på en hemsida. Du behövde inte längre skapa ett helt program i C, eller liknande, som i sin tur genererade HTML-kod för att exempelvis skapa en enkel gästbok. Nu kunde du enkelt lägga in PHP-kod direkt i HTML-koden som kördes i samband med att klienten frågar efter sidan. Det saknar även en standardiserad pakethanterare vilket gör att det är svårare att använda sig av många bibliotek. 

%PHP : Byggt för webben men är rätt gamalt, ingen pakethanterare, svårt att skala
%Python: Inte byggt för webben från början

\subsubsection{Python}

%http://flask.pocoo.org/
På senare tid har flera språk som inte är skapade för webben från början fått stöd för att göra hemsidor. Ett exempel är Python vilket alltså inte är byggt för webben till en början men ramverk som exempelvis Flask\cite{python_flask} gör det möjligt att starta en enkel webbserver som genererar HTML-kod. Till skillnad från PHP så separerar alltså Python logiken från själva HTML-koden och man skriver inte Python direkt i HTML-filerna. Programmeringsstilen är ganska lik Node.js och man använder sig ofta av liknande designmönster, men eftersom språket inte är skapat för webben från början kan det få problem med exempelvis prestanda när antalet användare ökar.

\subsubsection{Node.js}
%http://radar.oreilly.com/2011/07/what-is-node.html
Internet ändras hela tiden. Om man går tillbaka några år så bestod mest internet av statiska hemsidor med enkla gästböcker eller liknande. Idag består internet av sociala nätverk och sidor med dynamiskt innehåll som kan ha tusentals användare varje dag. Node.js är skapat med detta i åtanke. Det är inte skapat för att göra tunga beräkningar men däremot för att snabbt kunna distribuera information till flera användare samtidigt. En stor skillnad på internet idag gentemot när PHP skapades är att när man har laddat klart en hemsida så kan fortfarande saker uppdateras. Exempelvis kan man på många sidor direkt se när en ny kommentar skickas in eller som i detta projekt så ska flera personer kunna se när man kryssar i en checkruta. Detta är fullt möjligt i både PHP och Python också men Node.js är skapat just för att lösa sådana problem utan att prestanda ska bli lidande.

\subsection{Metod}

I takt med att vi kom igång med projektet så fick vi i gruppen hela tiden nya erfarenheter med Node.js. Vi har upptäckt både fördelar och nackdelar med arkitekturen som jag inte hade tänkt på tidigare. 

Utifrån vad jag har hört och sett från gruppen tillsammans med mina egna åsikter har jag tagit fram vad jag ser för fördelar och nackdelar med Node.js i detta projekt. Jag har även utifrån tidigare erfarenheter tagit med några potentiella fördelar och nackdelar som kan uppstå med denna arkitektur i längden.

Till sist har jag utifrån mina tidigare erfarenheter vägt fördelarna och nackdelarna gentemot hur det kan fungera i PHP och Python. 

%Skriv om:
%* Hur jag kom fram till fördelar/nackdelar
%* Hur jag kom fram till många externa bibliotek?
%* Såg min chans att prova något nytt?
% FRAMFÖRALLT: Samlat erfarenheter under projektet.

\subsection{Resultat}

Under denna rubrik presenteras resultatet av undersökningen. 

\subsubsection{Fördelar och nackdelar med Node.js}

Följande ska skrivas om i flytande text!
Fördelar:
\begin{itemize}
	\item Många bibliotek med npm
	\item Enkelt att börja med
	\item Samma språk som på klienten
	\item Simpelt med realtidsuppdateringar (socket.io)
\end{itemize}
Nackdelar:
\begin{itemize}
	\item Ibland svårt att strukturera koden
	\item Behövs inte speciellt tunga beräkningar för att prestandan ska bli lidande och man får därför lägga mycket logik på klienten. (exempelvis redigera-sidan)
	\item Inte skapat för relationsdatabaser vilket passar bättre till vissa av sakerna (kartoteket exempelvis)
	\item Non-blocking är svårt att greppa till en början
\end{itemize}

\subsubsection{Fördelar och nackdelar med externa bibliotek}

Följande ska skrivas om i flytande text!
Fördelar:
\begin{itemize}
	\item Ofta snygga och effektiva lösningar som hade tagit lång tid att skriva själv
	\item Går snabbt att nå resultat
	\item 
\end{itemize}
Nackdelar:
\begin{itemize}
	\item Konstiga buggar
	\item Förlitar sig på andra programmerare
	\item Svårt att sätta sig in i hur allt fungerar och känns ibland som saker sker av magi.
	\item Svårt att hålla reda på alla Licenser
\end{itemize}

%Skriv om:
%* Fördelar/nackdelar med Node.js
%* Fördelar/nackdelar med externa bibliotek
%* Konstiga buggar
%* Förlitar sig på andra
%* Svårt med nonblocking när man aldrig använt innan
%* Känns ibland som magi när man inte vet vad som händer i bakgrunden
%* Svårt att felsöka när man inte förstår vad som händer till 100\%.

\subsection{Diskussion}

\subsubsection{Resultat}

Ska skrivas:
\begin{itemize}
	\item Varför väger fördelarna över nackdelarna
	\item Vad kan man göra åt nackdelarna?
	\item Var Node.js ett bra val?
	\item Var det ett bra val att använda sig av mycket bibliotek eller skulle man varit mer konservativ?
	\item Skulle man kunna gjort något annorlunda?
\end{itemize}
%Skriv om: 
%* Varför fördelarna väger över nackdelarna
%* Vad man kan göra åt nackdelarna?
%* Var Node.js ett bra val?
%* Kan det finnas ett annat utgångsläge där ett alternativ varit bättre?

\subsubsection{Metod}

Ska skrivas:
\begin{itemize}
	\item Kunde jag gjort något annorlunda?
	\item Hade jag gjort något annorlunda om det funnits mer tid?
	\item Mer research om andra liknande projekt?
	\item Skulle jag tagit mer hänsyn till kunden?
\end{itemize}

%Skriv om:
%* Kunde jag gjort något annorlunda?
%* Varför använde jag just den metoden?
%* Tidsbrist? Tvingad till snabba beslut?
%* Önskemål från kund?
%* Kunde ha gjort mer research om andra liknande projekt

\subsection{Slutsatser}
Ska skrivas:
\begin{itemize}
	\item Var Node.js ett bra val?
	\item Kunde PHP eller Python ha varit bättre alternativ?
\end{itemize}
%Skriv om: 
%* Slutsatsen är att Node.js var ett bra val och fördelarna väger över nackdelarna. Jag tror inte heller att PHP hade varit ett bättre alternativ.

\subsection{Referenser}
\vspace{-9mm}
\begin{thebibliography}{9}

	\bibitem{php_history}
	\url{http://php.net/manual/en/history.php.php}\\
	Hämtad 2015-05-11.
	
	\bibitem{python_flask}
	\url{http://flask.pocoo.org/}\\
	Hämtad 2015-05-11.
	
	\bibitem{nodejs_whatis}
	\url{http://radar.oreilly.com/2011/07/what-is-node.html}\\
	Hämtad 2015-05-11.
	
\end{thebibliography}