\section{Automatiserade tester med Travis CI - Erik Malmberg}
\subsection{Inledning}
Den här enskilda utredningen är en del av kandidatrapporten i kursen TDDD77 vid Linköpings universitet.
Utredningen behandlar en del av utvecklingen av ett webb-baserat system för att underlätta förberedelser
inför operationer på sjukhusen i Östergötland. Systemet utvecklades på uppdrag av Region Östergötland.

\subsubsection{Syfte}
Syftet med den här enskilda delen av kandidatarbetet är att ge insikt i hur kontinuerlig integration och 
automatiserade tester kan användas för att effektivisera testandet i ett projekt som använder en agil 
utvecklingsmetod. Speciellt ska det undersökas hur väl det går att använda Travis CI tillsammans med 
ramverket Jasmine.

\subsubsection{Frågeställning}
De frågeställningar som ska besvaras i den här enskilda delen av rapporten är:

\begin{itemize}
\item Hur kan man använda Travis CI tillsammans med Jasmine för att testa en webbapplikation byggd på javascript och node.js.
\item Hur många tester hinner Travis CI köra på en sekund?
\item Vilka typer av tester är svåra att utföra?
\end{itemize}

I svaret på den andra frågeställningen ska testfallen specifieras noggrant 
så att svaret inte blir tvetydigt.

\subsubsection{Avgränsningar}
Inga undersökningar kommer att utföras om hur andra lösningar än Travis CI kan användas för kontinuerlig 
integration. De testfall som kommer användas kommer uteslutande att vara skrivna med ramverket Jasmine.

\subsection{Teori}
Här beskrivs den teori som är nödvändig för att förstå rapporten.

\subsubsection{Vattenfallsmodellen}
I vattenfallsmodellen genomförs all integration och alla tester efter att implementeringen är slutförd. 
Om ett problem då identifieras under integrationen så är det krångligt att gå tillbaka och åtgärda problemet. 
Det kan leda till förseningar av projektet.
Om felet som upptäcks är så allvarligt att en betydande omdesign måste ske så
kommer utvecklingen i stort sett att börja om från början och man kan räkna 
med en hundraprocentig ökning av budgeten, 
både vad gäller pengar och tid \cite{Royce}.

\subsubsection{Kontinuerlig integration och automatiserade tester}
Kontinuerlig integration kan leda till att problemen identifieras tidigare i 
utvecklingsprocessen. Problemen blir då lättare att åtgärda. Automatiserade tester kan effektivisera 
testprocessen och det finns många tillgängliga lösningar för att köra automatiserade
tester \cite{Karlsson}.
Några av de vanligaste är Travis CI, Codeship och Drone.

\subsubsection{Travis CI}
Travis CI är en webb-baserad tjänst för att köra automatiserade enhetstester och integrationstester
på projekt som finns på GitHub. Travis CI är byggt på öppen källkod och är gratis att använda. 
Tjänsten har stöd för många olika programmeringsspråk, men det som är relevant för innehållet i den här rapporten
är javascript med node.js. För att konfigurera Travis CI används filen .travis.yml som placeras i det aktuella
projektets repository på GitHub.

\subsubsection{Javascript}
Javascript är ett programmeringsspråk som i första hand används på klientsidan på webbsidor.
Javascript exekveras av webbläsaren och arbetar mot ett gränssnitt som heter Document Object Model (DOM).

\subsubsection{Node.js}
Node.js är en runtime environment för internetapplikationer. Det kan till exempel användas för att skapa webbservrar.
Node.js är baserat på öppen källkod och det är enkelt att lägga till nya moduler för att anpassa det system man vill
använda. För att lägga till nya moduler används node package manager (npm).

\subsubsection{Jasmine}
Jasmine är ett ramverk för testning av Javascript. 
Den node-modul som används är grunt-contrib-jasmine som använder task runnern Grunt 
för att köra testfall som skrivits med Jasmine.
Grunt kunfigureras med filen Gruntfile.js.

\subsection{Metod}
Arbetet inleddes genom att Travis CI kopplades till projektets repository på GitHub.
Kopplingen utfördes
genom att administratören för repositoryn loggade in på travis-ci.org med 
sitt GitHub-konto och aktiverade
en webhook för repositoryn.\\

Inställningarna för Travis CI konfigurerades med filen .travis.yml i projektets
repository. Språket valdes till
javascript med node.js med inställningen: \emph{language: node\textunderscore js}.
Versionen av node.js valdes till version 0.10
med inställningen: \emph{node\textunderscore js: "0.10"}.\\

De nödvändiga node-modulerna installerades med hjälp av node package manager (npm).
Grunt installerades
med kommandot: \emph{npm install -g grunt-cli}. Grunt-contrib-jasmine installerades med kommandot: 
\emph{npm install grunt-contrib-jasmine}.\\

Task runnern Grunt konfigurerades med filen Gruntfile.js i projektets repository.
En task för Jasmine skapades med
inställningen: \emph{grunt.loadNpmTasks('grunt-contrib-jasmine');}.
Eftersom Travis CI använder npm för att starta 
testerna så definierades testskriptet för npm med raden
\emph{''test'': ''grunt jasmine --verbose''} i filen package.json 
i projektets repository.

\subsection{Resultat}
\subsection{Diskussion}
\subsubsection{Resultat}
\subsubsection{Metod}
\subsection{Slutsatser}
\subsection{Referenser}
\begin{thebibliography}{9}

\bibitem{Royce}
W.W. Royce, ''Managing the development of large software systems,''
\textit{Proceedings of IEEE WESCON}, pp. 2, aug, 1970.
[Online].
Tillgänglig (nytryckt med annan sidnumrering):
\url{http://www.cs.umd.edu/class/spring2003/cmsc838p/Process/waterfall.pdf}.
[Hämtad april 28, 2015].

\bibitem{Karlsson}
O. Karlsson, ''Automatiserad testning av webbapplikationer,''
Linköpings univ., Linköping, Sverige, 2014, pp. 43.
[Online]. 
Tillgänglig: 
\url{http://www.diva-portal.org/smash/get/diva2:727654/FULLTEXT01.pdf}.
[Hämtad april 19, 2015].

\end{thebibliography}
