\section{Albert Karlsson}
\subsection{Inledning}

\subsubsection{Syfte}
Syftet med denna enskilda utredningen är att undersöka olika kvalitetsaspekter då ett mindre projekt utförs av en liten grupp utvecklare.
\subsubsection{Frågeställning}
\begin{itemize}
\item Finns det några kvalitetshöjande processer som enkelt kan implementeras utan mycket resurser och kunskap?
\item Behövs en kvalitetssamordnare i små grupper? Kan inte alla andra dela på detta arbetet?

\end{itemize}
\subsubsection{Avgränsningar}
Rapporten har sin utgångspunkt i hur kvalitetsarbetet har gått till i detta projekt och ska inte ses som ett allmänt tillvägagångsätt. 
\subsection{Bakgrund}
I små projektgrupper blir kvalité ofta något som får stå åt sidan för andra saker som gruppen anser vara mer viktiga, så som implementering av nya funktioner. Det är ofta svårt att motivera utvecklare till att jobba med kvalitet eftersom arbete med kvalité sällan visar direkta resultat. 
\subsection{Teori}
\subsection{Metod}
Mycket information kommer hämtas från andras erfarenheter inom kvalitetsarbete i små grupper. En sammanställning kommer sedan göras och det som många anser vara viktigt kommer sedan testas i projektgruppen.    
\subsection{Resultat}
\subsection{Diskussion}
\subsubsection{Resultat}
\subsubsection{Metod}
\subsection{Slutsatser}
