\section{Vidareutveckling av applikation för Region Östergötland - Albert Karlsson}
\subsection{Inledning}
Denna del i rapporten behandlar vad som ska utredas och varför.
\subsubsection{Syfte}
Syftet med denna enskilda utredningen är att underlätta för fortsatt utveckling av webbapplikationen som projektgruppen har skapat. Alla delar i applikationen får inte användas i Region Östergötlands intranät, vilket leder till att en del måste bytas ut.
\subsubsection{Frågeställning}
\begin{itemize}
\item Vad används Keystone till i applikationen?
\item Vad behövs för att ersätta Keystone?
\item Vad behöver ändras för att byta ut databasen från MongoDB till MSSQL?
\item Vilka moduler eller bibliotek i applikationen kräver en licens för kommersiell användning?

\end{itemize}
\subsubsection{Avgränsningar}
Denna rapport gäller endast för vidareutveckling för användning av Region Östergötland. Andra användare kan ha andra krav på applikationen som leder till att denna rapport är ofullständig eller felaktig. 
\subsection{Bakgrund}
Region Östergötland ska ta över arbetet med utvecklingen av webbapplikationen efter att projektgruppen slutfört sitt arbete. För att applikationen ska kunna tas i bruk på riktigt så måste databasen bytas ut till MSSQL då Region Östergötland inte tillåter MongoDB på sina servrar. Då utvecklingen av applikationen har fortgått har Keystone fått en mindre och mindre roll i applikationen. En av Keystones största fördelar är dess enkla och smidiga administratörssystem. Detta används inte alls i applikationen längre och då databsen ska bytas till en annan typ försvinner också en annan stor del av Keystone. Detta ledde till tankar om att Keystone kanske skulle kunna bytas ut mot egenskriven kod eller mindre och mer lättförståliga moduler utan jättemycket arbete.
\subsection{Teori}
En beskrivning och förklaring för många av modulerna som kommer tas upp finns att läsa i avsnitt 3.
\subsubsection{MSSQL}
MSSQL är en databashanterare från Microsoft som använder SQL som frågespråk.

\subsubsection{npm}
npm är en pakethanterare för Node.js som hanterar open-source paket. Det finns även en tillhörande hemsida för npm där teknisk dokumentation med mera för olika paket kan läsas. 


\subsection{Metod}
För att få en bättre förståelse för Keystones roll i applikationen så läses först och främst Keystones tekniska dokumentation. En ny installation av Keystone görs för att kunna jämföra med projektkoden och få fram vilka komponenter som kommer från Keystone. Keystone är också beroende av många moduler. Dessa moduler kommer utvärderas för att se om de skulle fungera bra i en version av applikationen utan Keystone. 

Information om olika moduler kommer tas från respektive utgivares hemsida eller, i de fall de inte finns, npmjs.com eller github.
express MIT

\subsection{Resultat}
\subsection{Diskussion}
\subsubsection{Resultat}
\subsubsection{Metod}
\subsection{Slutsatser}
