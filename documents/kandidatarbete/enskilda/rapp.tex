\section{Kartoteket - Daniel Rapp}
\subsection{Inledning}
Idag är information om Region Östergötlands operationsartiklar, så som priserna och
placeringen i lagret på tandborstar, tandkräm, handskar och annan
medicinsk utrustning, hanterat av ett internt system.
Detta system kallas ett "\textit{kartotek}".


\subsubsection{Syfte}
Det största problemet med det nuvarande kartoteket är
att det inte är integrerat med systemet för att hantera handböcker.

I dagsläget, om en artikel slutas säljas eller Region Östergötland
väljer att inte köpa in en viss artikel längre så tar de bort artikeln
från kartoteket. Problemet som då uppstår är att detta inte reflekteras
i handböckerna. Så om t.ex. en "\textit{Oral-B Pro 600 CrossAction}"\ tandborste används i
en "\textit{Laparoskopisk sigmoideumresektion}", och tandborsten slutas säljas
så tas den bort från kartoteket, men eftersom handboken för operationen inte är
kopplad till kartoteket så uppdateras det inte att denna artikel inte längre finns
i lagret.

Syftet med denna del av systemet är att lösa detta problem genom att
integrera systemet som hanterar handböcker tillsammans med ett nytt kartotek,
allesammans byggt på webben. När en artikel ändras eller tas bort i kartoteket
så ändras den även i alla handböcker för operationer som kräver denna artikel.


%Syftet med denna del av projektet är att..

%Syftet med projektet är att bygga en web-baserad, mobilanpassad, prototyp som ersätter, och förbättrar, det pappersbaserade systemet.


\subsubsection{Frågeställning}
Frågeställningar:
\begin{itemize}
  %\item Kan man implementera ett kartotekssystem som uppfyller kundens önskemål?
  %\item Går det att implementera ett kartotekssystem som 
  \item Går det att integrera systemet för handböcker med kartoteket utan att förlora funktionalitet?
\end{itemize}


\subsubsection{Avgränsningar}
Förutom ett förbättrat avcheckningsssytem så är Region Östergötland också i behov av
ett bättre system för att hantera deras lager på ett mer automatiserat sätt. %, med t.ex. en
%scanner för att scanna av artiklar istället för manuellt checka av dem.
Bland annat
så skulle de behöva ett system som låter dem checka in vilka varor från lagret de hämtat
ut, istället för att checka av manuellt, vilket kan vara felbenäget.

Vi har dock valt att avgränsa oss från att bygga denna lösning, på grund av tidsbrister.


\subsection{Bakgrund}


\subsection{Teori}
%Agilt

\subsection{Metod}
\subsection{Resultat}
\subsection{Diskussion}
\subsubsection{Resultat}
\subsubsection{Metod}
\subsection{Slutsatser}
\subsection{Referenser}
\vspace{-9mm}
\renewcommand{\refname}{}
\begin{thebibliography}{9}

\end{thebibliography}
