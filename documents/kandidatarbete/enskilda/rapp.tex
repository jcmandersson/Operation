\section{Daniel Rapp}
\subsection{Inledning}
Idag, när Region Östergötland tar hand om en patient i behov av en operation, så använder de sig av pappersbaserade handböcker och checklistor för att
förbereda operationen och ta fram rätt operationsutrustning.

\subsubsection{Syfte}
Syftet med projektet är att bygga en web-baserad, mobilanpassad, prototyp som ersätter, och förbättrar, det pappersbaserade systemet.


\subsubsection{Frågeställning}
Frågeställningar:
\begin{itemize}
  \item Kan man implementera ett kartotekssystem som uppfyller kundens önskemål?
\end{itemize}


\subsubsection{Avgränsningar}
Förutom ett förbättrat avcheckningsssytem så är Region Östergötland också i behov av
ett bättre system för att hantera deras lager på ett mer automatiserat sätt%, med t.ex. en
%scanner för att scanna av artiklar istället för manuellt checka av dem.
Bland annat
så skulle de behöva ett system som låter dem checka in vilka varor från lagret de hämtat
ut, istället för att checka av manuellt, vilket kan vara felbenäget.

Vi har dock valt att avgränsa oss från att bygga denna lösning, på grund av tidsbrister.


\subsection{Bakgrund}


\subsection{Teori}
Agilt

\subsection{Metod}
\subsection{Resultat}
\subsection{Diskussion}
\subsubsection{Resultat}
\subsubsection{Metod}
\subsection{Slutsatser}
