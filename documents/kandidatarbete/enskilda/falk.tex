\section{Titel}
\subsection{Inledning}
Jag har i detta projekt haft rollen som analysansvarig vilket innebär en analys av kundens behov och framställning av krav utifrån dessa. Rapporten beskriver hur kravframställningen gått till i detta projekt och hur prototyper har använts för att kommunicera idéer.
\subsubsection{Syfte}
Syftet med denna del av rapporten är att ta reda på hur man kan sammla in krav som kunden har på produkten. Fokus ligger på användnigen av prototyper men utgångspunkt i detta projekt.
\subsubsection{Frågeställning}
Hur kan man samla in krav som tillfredställer kundens behov?
Hur kan man arbeta med prototyper för att framställa krav?

\subsubsection{Avgränsningar}
Rapporten har sin utgångspunkt i hur kravframställningen gått till i detta projekt. Metoden ska inte ses som ett allmänt tillvägagångssätt.
\subsection{Bakgrund}
Ett projekt faller ofta på grund av dålig kravframställning(någon referens). 
\subsection{Teori}
\subsection{Metod}
\subsection{Resultat}
\subsection{Diskussion}
\subsubsection{Resultat}
\subsubsection{Metod}
\subsection{Slutsatser}
