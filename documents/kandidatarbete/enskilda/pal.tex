\section{Pål Kastman}
\subsection{Inledning}
I detta projekt så har vi valt att inledningsvis skriva dokumentationen i Googles ordbehandlingsprogram Google Docs för att 
i ett senare skede då det var dags att påbörja denna rapport gå över till att använda typsättningssystemet Latex. 

Vi valde att göra på detta sätt för att så snabbt som möjligt komma igång, då man bland annat i Google Docs kan se live 
vad andra skriver.

\subsubsection{Syfte}
Syftet med denna individuella del är att undersöka funktionaliteten i Latex och väga fördelar mot nackdelar.

\subsubsection{Frågeställning}
\begin{itemize}
\item Vad finns det för begränsningar i Latex
\item Kommer det att vara en fördel eller en nackdel för flera gruppmedlemmar att jobba samtidigt i samma delar av rapporten.
\end{itemize}

\subsubsection{Avgränsningar}
Denna rapport kommer att avgränsas för att endast jämföra Latex, Microsoft Office och Google Docs, detta för att 
inte behöva jämföra alla olika ordbenhandlingsprogram.

\subsection{Bakgrund}
Dokumentering är någonting som är viktigt att göra när man arbetar i projektform, dels för egen del utifall man behöver 
gå tillbaka och se vad som gjorts, men även för andras skull ifall man kanske får en ny medarbetare som ska integreras i 
projektet. En fråga som man alltid behöver besvara är i vilket ordbehandlingsprogram dokumentationen skall skrivas. Det populäraste 
alternativet kan tänkas vara Microsoft Word, vilket har funnits sedan 1983\cite{word_ursprung}.
\newline
Ett annat alternativ som blir allt vanligare är Google Docs.
\newline
Ett tredje och inte lika vanligt alternativ är Latex, vilket inte är ett ordbehandlingsprogram utan istället ett märkspråk
 så som t.ex. HTML, där man istället för med ett grafiskt gränssnitt formaterar sin text,sätter sin text inom speciella taggar så att när koden senare kompileras får rätt utseende.

Latex är egentligen bara en uppsättning makron skrivna för språket Tex, vilket skapades av den amerikanske matimatikern Donald Knuth 1978\cite{donald_knuth}. I början av 80-talet så vidareutvecklade Leslie Lamport Tex med hjälp av dess makrospråk till det som 
idag är Latex \cite{leslie_lamport}.

Vad det gäller min egen bakgrund i Latex så hade jag när detta projektet påbörjades endast använt det i ett tidigare projekt, under det projektet
blev jag dock inte så insatt i Latex utan fyllde endast i material i de redan färdiga mallarna vi hade.

Under kursens gång har jag dock skrivit en laborationsrapport i Latex och därigenom skaffat mig lite mer erfarenhet.


\subsection{Teori}
När Microsoft utvecklade Word på 80-talet så ville de naturligtvis att dåtidens datorer skulle klara av att ladda 
dokument utan att det skulle vara alltför prestandakrävande, därför konstruerade man dess filformat (.doc) binärt och gjorde konstruktionen 
väldigt komplex. Detta medförde att bara personer som var riktigt insatta i detta filformat klarade att ändra direkt i dessa filer.
All textformatering i Word görs genom att använda de olika snabbknapparna i verktygsfälten högst upp, det finns även tangentbordskommandon man kan
använda om man vill.

Latex är i motsats till Word och Google docs inget ordbehandlingsprogram, utan istället ett märkspråk liksom HTML. När man i Word och Google docs 
ändrar textformateringen genom ett grafiskt gränssnitt så markerar man istället sin text i Latex med taggar, som senare när man kompilerar koden ger
det önskade utseendet. Detta gör att man som användare behöver bry sig mindre om utseendet av dokumentet och kan istället fokusera på innehållet.


\begin{figure}[htbp]
\begin{center}
\includegraphics[scale=0.4]{wordvslatex.png}
\caption{Textformattering i det grafiska gränssnittet i word till vänster, respektive latex-taggar till höger}
\label{fig:msword}
\end{center}
\end{figure}


\subsection{Metod}
Vi har till en början valt att under projektets gång arbeta i endast en fil för den gemensamma delen av kandidatprojektet. Vad det gäller 
de individuella delarna så har vi valt att lägga dessa i separata filer och sedan importera dessa till den gemensamma. Vi 
sparar alla dokumentfiler i samma repository på github som vi har källkoden.

Genom att göra på detta sätt så kan vi garantera att det inte kommer att uppstå några konflikter i de individuella delarna.
Det som vi inte vet, är huruvuda det kommer att uppstå problem då alla gruppmedlemmar skall skriva på den gemensamma delen.

\subsection{Resultat}
\subsection{Diskussion}
\subsubsection{Resultat}
\subsubsection{Metod}
\subsection{Slutsatser}
\subsection{Referenser}
\begin{thebibliography}{9}
\bibitem{word_ursprung}
\url{http://ia801406.us.archive.org/21/items/A\_History\_of\_the\_Personal\_Computer/eBook12.pdf}.\\
 Sida 11-12. Hämtad 2015-04-20.
\bibitem{donald_knuth}
\url{https://gcc.gnu.org/ml/java/1999-q2/msg00419.html}.\\
 Hämtad 2015-04-20.
\bibitem{leslie_lamport}
\url{http://research.microsoft.com/en-us/um/people/lamport/pubs/pubs.html#latex}.\\
 Hämtad 2015-04-20.
 
\end{thebibliography}
