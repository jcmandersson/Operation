\section{Latex - Pål Kastman}
\subsection{Inledning}
I detta projekt har jag haft rollen som dokumentansvarig. En del av det arbete jag har utfört i denna roll har gått ut på att se till att dokumenten ser ordentliga ut och att hitta bra lösningar för dokumentation.

I denna individuella del av rapporten så undersöker jag hur det är att arbeta med Latex i denna typ av projektform jämfört med andra orbehandlingsprogram.

\subsubsection{Syfte}
Syftet med denna individuella del är att undersöka funktionaliteten i Latex och väga fördelar mot nackdelar.

\subsubsection{Frågeställning}
\begin{itemize}
\item Uppmuntrar Latex till en viss typ av samarbete jämfört med andra ordbehandlingsprogram?
\item Kommer det att vara en fördel eller en nackdel för flera gruppmedlemmar att jobba samtidigt i samma delar av rapporten?
\end{itemize}

\subsubsection{Avgränsningar}
Denna rapport kommer att avgränsas för att endast jämföra Latex och Google Docs, detta för att inte behöva jämföra alla olika ordbenhandlingsprogram.

\subsection{Bakgrund}
Vid utveckling av en produkt i projektform hamnar ofta utvecklingen av produkten i fokus och dokumenteringen blir utdaterad. Dokumentering är dock viktig att ha av flera anledningar, till exempel om man behöver utföra underhåll av, eller vill återanvända, ett system \cite{m_spichkova}.

Ett alternativ som blir allt vanligare är Google Docs, vilket är ett webbaserat ordbehandlingsprogram som lanserades av Google 2006 \cite{docs_launch} efter att man hade köpt upp företaget Upstartle \cite{upstartle}. En stor fördel med detta program är att det är gratis så länge man har ett konto hos Google.

Ett annat alternativ är Latex, vilket egentligen bara är en uppsättning makron skrivna för språket Tex, vilket skapades av den amerikanske matematikern Donald Knuth 1978 \cite{donald_knuth}. I början av 80-talet vidareutvecklade Leslie Lamport Tex med hjälp av dess makrospråk till det som 
idag är Latex \cite{leslie_lamport}.

Vad det gäller min egen bakgrund i Latex så hade jag när detta projekt påbörjades endast använt det i ett tidigare projekt. Under det projektet blev jag dock inte så insatt i Latex utan fyllde endast i material i de redan färdiga mallarna vi hade.

Under kursens gång har jag dock skrivit en laborationsrapport i Latex och därigenom skaffat mig lite mer erfarenhet.


\subsection{Teori}
I denna del behandlas en del teori om Google Docs och Latex.
\subsubsection{Google Docs}
Google Docs använder sig av molnet för att spara filer och man behöver därför inte oroa sig över säkerhetskopiering. Textformateringen görs främst genom att använda det grafiska gränssnittet, men kan även göras genom tangentbordskommandon. Det finns ett antal inbyggda funktioner för att t.ex. importera bilder, skapa tabeller eller kommentera texten.

En stor fördel med Google Docs är att dokumenten live-uppdateras hela tiden och att man på så sätt kan arbete utföras samtidigt av flera personer i samma dokument.

\subsubsection{Latex}
Latex är i motsats till Google Docs inget ordbehandlingsprogram, utan istället ett märkspråk liksom HTML. När man i Google Docs ändrar textformateringen genom ett grafiskt gränssnitt så markerar man istället sin text med taggar, som senare när man kompilerar koden ger det önskade utseendet. 

Detta gör att man som användare behöver bry sig mindre om utseendet av dokumenten och istället kan fokusera på innehållet. Eftersom Latex inte är något program i sig utan ett programspråk behöver man något program att redigera koden i. Detta kan göras i vilken textredigerare som helst, men det finns även program som kan kompilera koden åt användaren så att man kan se direkt vilka ändringar som görs. Det har på senare år dykt upp flera webbsidor där man kan skriva Latexdokument live, tillsammans med andra användare (ungefär som Google Docs). Exempel på sådana sidor är Overleaf \cite{overleaf} och Authorea \cite{authorea}. Latex har ett stort antal inbyggda kommandon och det är mycket enkelt att t.ex. skriva formler vilket har gjort Latex till standard inom den vetenskapliga sektorn \cite{latex_standard}.

\subsection{Metod}
I detta projekt har vi valt att inledningsvis skriva dokumentationen i Google Docs för att i ett senare skede, då det var dags att påbörja denna kandidatrapport, gå över till att använda Latex. Vi valde att göra på detta sätt för att så snabbt som möjligt komma igång då det är lätt  för flera personer att arbeta samtidigt i samma dokument.

I denna rapport skrev vi inledningsvis den gemensamma delen i samma fil, och de indivuella delarna i separata filer som vi sedan importerade till den gemensamma. I ett senare skede lade vi även ut vissa delar av den gemensamma rapporten i separata filer. Alla Latexfiler har vi haft i samma repository som källkoden.

Varje dokument har en dokumenthistorik där tanken är att när man gör en ändring så ger man dokumentet en ny version och skriver vilka ändringar man gjort.

Vi använde IEEE-standarden för referenser.

\subsection{Resultat}
I denna del presenteras resultatet -- det är givetvis inte bara resultatet av mitt eget arbete utan projektgruppens arbete som en helhet.

\subsubsection{Dokumentstandard}
När vi i projektets inledning använde Google Docs var det mycket svårare att hålla sig till någon sorts dokumentstandard, där man bl.a. använder samma typsnitt på rubriker och text. Även om alla projektmedlemmar försökte att hålla sig till den standard som vi satte upp blev det ändå någon del av dokumenten skiljde sig från resten när vi redigerade samtidigt i dem.

Detta fungerade bättre i Latex, men inte helt utan hinder, då det även här uppstod problem med att olika personer gjorde på olika sätt.

\subsubsection{Revisionshantering}
Under projektets gång har vi försökt att hålla isär olika revisioner av dokument så att man i ett senare skede kan gå tillbaka och se vilka ändringar som har gjort mellan iterationerna.
Detta har vi dock haft en del problem med i Google Docs då dokumenten automatiskt sparas löpande när man skriver och det inte på något smidigt sätt går att se historiken. Detta löste vi genom att göra en ny kopia på en fil när man ska skapar en ny revision.

I kandidatrapporten var inte detta ett lika stort problem då vi, som tidigare nämnt, använde oss utav revisionshanteringssystemet git. Detta gjorde att vi utan problem kunde gå tillbaka och se vad som hade ändrat sig i dokumentet. Dock har detta inte heller varit helt problemfritt då vi ibland fått mergekonflikter när vi laddat upp dokumentext. Detta beror på att om flera personer ändrar på samma rader i ett dokument så kan git inte avgöra vilken text som ska sparas och då måste användarna lösa detta tillsammans, vilket inte alltid varit så enkelt om man sitter och skriver på olika ställen.

Ett annat problem har varit att om en gruppmedlem har lagt till en bild så måste denne komma ihåg att lägga till bilden i git så att även denna laddas upp. Om detta inte har gjorts när någon annan har pullat koden har bilden saknats och det har inte varit möjligt att kompilera dokumentet.

\subsubsection{Kommentering}
Kommentering av varandras texter t.ex. vid korrekturläsning har skett smidigt i Google Docs med den inbyggda kommenteringsfunktion. Den projektmedlem som varit ansvarig för texten har sedan fått välja mellan att lösa problemet eller att ge ett svar på kommentaren.

Kommentering av texter i Latex har vi gjort på github istället för i dokumenten.

\subsubsection{Referenser}
Referenser har varit lite av ett problem, eftersom vi hade bestämt att använda IEEE-standarden men det i Google Docs inte finns något inbyggt system för detta. Detta gjorde att vi var tvungna lägga till referenserna manuellt.

I Latex fanns det däremot ett antal sätt att göra detta på, vilket istället ledde till att vi valde ett sätt som senare skulle visa sig vara ganska omständigt att använda, vilket ledde till att formateringen av referenserna blev onödigt komplicerat.

\subsection{Diskussion}
Här diskuteras resultatet och metoden

\subsubsection{Resultat}
Latex kan vara ett väldigt trevligt sätt att skriva dokument på även om ett det kan ta en hel del tid innan man har vänjt sig helt. Man får en bra hjälp med att hålla samma dokumentstandard. Vad som dock kan vara värt att tänka på är att det finns en hel del valmöjligheter, och om man vill jobba med Latex i en större grupp bör man ha klart för sig att det kan vara en bra idé att prata ihop sig för att avgöra vad man får och inte får göra.

Man kan ibland känna sig väldigt begränsad i Google Docs, t.ex. kan man inte på något sätt välja på vilken sida sidnumreringen ska börja. Dock tycker jag ändå att detta kan vara till en fördel, för om det endast finns en lösning till ett problem slipper man leta efter det bästa sättet, vilket ibland blir fallet i Latex.

Kommentering av dokument är någonting som har varit svårt att lösa, kommenteringsfunktionen i Google Docs fungerar jättebra så länge man vet vilken text man ska kommentera. Så länge man kommenterar sin egen text är det inga problem, däremot om man vill kommentera någonting som tidigare har skrivits av en annan gruppmedlem så kan det vara väldigt svårt att veta exakt vad denna medlem har skrivit, särskilt om denna om inte revisionshanteringen sköts på rätt sätt.

\subsubsection{Metod}
Uppdelningen som vi gjorde i detta projekt med att inleda dokumenteringen i Google Docs och sedan gå över till Latex tycker jag har fungerat väldigt bra. Vi fick den tiden vi behövde för att sätta oss in i Latex och vi slapp onödig stress. Om alla involverade i ett projekt redan är insatta i Latex tror jag absolut att det går minst lika bra att använda Latex från början till slut.

Dock bör man även tänka på hur många man är i ett projekt då man med största förmodan behöver använda ett revisionshanteringssystem för att lagra dokumenten, och desto fler personer som behöver skriva i samma dokument samtidigt, desto större problem kommer man att få med konflikter. Är fallet att man har en klar uppdelning även om vem som skriver om vad och i vilka filer så ser jag dock inte detta som något problem alls och man bör inte stöta på några problem.

Även om det finns väldigt många fördelar med Latex och dokumenten blir väldigt prydliga så kan jag ändå se användningsområden för Google Docs då man t.ex. vill skriva interna dokument som inte behöver vara lika iögonfallande eller formella då den inbyggda kommentarsfunktionen är praktisk och fungerar väldigt bra.

I några av webbalternativen för Latex så finns det även en kommenteringsfunktioner. Dessa alternativ tror verkligen är någonting för framtiden då jag tycker att man får det bästa från flera världar.

\subsection{Slutsatser}
Även om Latex hjälper till att hålla en standard bör man ändå se till att undersöka vilka metoder som är lämpligast att använda för att lösa ens problem, då det i Latex ofta finns många olika sätt att lösa samma saker på. Någonting som vid en första anblick ser ut att vara det bästa sättet att lösa sitt problem på kanske senare visar sig vara rätt otympligt att handskas med.

Det kan vara en nackdel för flera gruppmedlemmar att arbeta samtidigt i samma delar av ett Latexdokument. Detta går till en viss utsträckning att lösa genom att dela upp dokumentet i flera separata filer. Man kommer dock inte helt ifrån problemet om flera personer behöver redigera exakt samma text -- i dessa fall stöter man på problem även om man använder sig av Google Docs.