\section{Pål Kastman}
\subsection{Inledning}
I detta projekt så har vi valt att inledningsvis skriva dokumentationen i Googles ordbehandlingsprogram Google Docs för att i ett senare skede, då det var dags att påbörja denna rapport gå över till att använda typsättningssystemet Latex. 

Vi valde att göra på detta sätt för att så snabbt som möjligt komma igång, då man bland annat i Google Docs kan se live 
vad andra skriver.

\subsubsection{Syfte}
Syftet med denna individuella del är att undersöka funktionaliteten i Latex och väga fördelar mot nackdelar.

\subsubsection{Frågeställning}
\begin{itemize}
\item Vad finns det för begränsningar i Latex
\item Kommer det att vara en fördel eller en nackdel för flera gruppmedlemmar att jobba samtidigt i samma delar av rapporten.
\end{itemize}

\subsubsection{Avgränsningar}
Denna rapport kommer att avgränsas för att endast jämföra Latex, Microsoft Office och Google Docs, detta för att inte behöva jämföra alla olika ordbenhandlingsprogram.

\subsection{Bakgrund}
Dokumentering är någonting som är viktigt att göra när man arbetar i projektform, dels för egen del utifall man behöver gå tillbaka och se vad som gjorts, men även för andras skull ifall man kanske får en ny medarbetare som ska integreras i 
projektet. En fråga som man alltid behöver besvara är i vilket ordbehandlingsprogram dokumentationen skall skrivas. Det populäraste alternativet kan tänkas vara Microsoft Word, vilket har funnits sedan 1983 \cite{word_ursprung}.

Ett annat alternativ som blir allt vanligare är Google Docs, vilket är ett web-baserat ordbehandlingsprogram som lanserades av Google 2006 \cite{docs_launch} efter att man hade köpt upp företaget Upstartle \cite{upstartle}.

Ett tredje och inte lika vanligt alternativ är Latex, vilket inte är ett ordbehandlingsprogram utan istället ett märkspråk så som t.ex. HTML, där man istället för med ett grafiskt gränssnitt formaterar sin text,sätter sin text inom speciella taggar så att när koden senare kompileras får rätt utseende.

Latex är egentligen bara en uppsättning makron skrivna för språket Tex, vilket skapades av den amerikanske matimatikern Donald Knuth 1978 \cite{donald_knuth}. I början av 80-talet så vidareutvecklade Leslie Lamport Tex med hjälp av dess makrospråk till det som 
idag är Latex \cite{leslie_lamport}.

Vad det gäller min egen bakgrund i Latex så hade jag när detta projektet påbörjades endast använt det i ett tidigare projekt, under det projektet blev jag dock inte så insatt i Latex utan fyllde endast i material i de redan färdiga mallarna vi hade.

Under kursens gång har jag dock skrivit en laborationsrapport i Latex och därigenom skaffat mig lite mer erfarenhet.


\subsection{Teori}
När Microsoft utvecklade Word på 80-talet så ville de naturligtvis att dåtidens datorer skulle klara av att ladda dokument utan att det skulle vara alltför prestandakrävande, därför konstruerade man dess filformat (.doc) binärt och gjorde konstruktionen väldigt komplex. Detta medförde att bara personer som var riktigt insatta i detta filformat klarade att ändra direkt i dessa filer. Google docs däremot använder sig av "molnet" för att spara filer och man behöver därför aldrig oroa sig över säkerhetskopiering. Textformatering i båda dessa program görs främst genom att använda det grafiska gränssnittet, men kan även göras genom tangentbordskommandon.

Latex är i motsats till Word och Google docs inget ordbehandlingsprogram, utan istället ett märkspråk liksom HTML. När man i Word och Google docs ändrar textformateringen genom ett grafiskt gränssnitt så markerar man istället sin text med taggar, som senare när man kompilerar koden ger det önskade utseendet. 

%exempelkod vore bra här!
	
Detta gör att man som användare behöver bry sig mindre om utseendet av dokumentet och kan istället fokusera på innehållet. Eftersom Latex inte är något program i sig utan mer ett programspråk så behöver man något program att redigera koden i, detta kan göras i vilken textredigerare som helst, men det finns även program som kan kompilera koden åt användaren så att man kan se direkt vilka ändringar som görs. Det har på senare år dykt upp flera webbsidor där man kan skriva Latexdokument live, tillsammans med andra användare (ungefär som Google docs). Exempel på sådana sidor kan vara Overleaf \cite{overleaf}, eller Authorea \cite{authorea}. Latex har väldigt många inbyggda kommandon och det är väldigt lätt att t.ex. skriva formler vilket har gjort Latex till standard inom den vetenskapliga sektorn \cite{latex_standard}.


\subsection{Metod}
Vi har till en början valt att under projektets gång arbeta i endast en fil för den gemensamma delen av kandidatprojektet. Vad det gäller de individuella delarna så har vi valt att lägga dessa i separata filer och sedan importera dessa till den gemensamma. Vi sparar alla dokumentfiler i samma repository på github som vi har källkoden.

Genom att göra på detta sätt så kan vi garantera att det inte kommer att uppstå några konflikter i de individuella delarna.Det som vi inte vet, är huruvuda det kommer att uppstå problem då alla gruppmedlemmar skall skriva på den gemensamma delen.

\subsection{Resultat}
När vi i projektets inledning använde Google docs så märkte vi att det var väldigt svårt att hålla sig till någon sorts dokumentstandard där man t.ex. använder samma typsnitt på rubriker och text. Även om alla projektmedlemmar har försökt att hålla sig till den standard vi satt upp så blev det ändå ibland att någon del av dokumenten blev annorlunda när vi redigerade samtidigt i dessa. Detta fungerade mycket bättre i Latex när vi skrev kandidatrapporten, latex hjälper verkligen användaren till att hela tiden hålla samma standard i dokumenten.

Vi har under projektet försökt att hålla isär på alla olika revisioner av dokument så att man senare kan gå tillbaka och se vilka ändringar som har gjorts från en iteration till en annan, detta har varit svårt att upprätthålla med Google docs då dessa dokument hela tiden sparas kontinuerligt och det inte finns något sätt att gå tillbaka i ett dokument för att se vilka ändringar som gjorts. Detta har gjort att när vi ändrat i dokument, behövt spara undan en kopia av den föregående revisionen vilket inte alltid har gjorts. I kandidatrapporten däremot har vi använt oss utav revisionshanteringsprogrammet git och på så sätt lätt kunnat revisionshantera även våra dokument och enkelt kunnat se vilka ändringar som gjorts. Detta har även hjälpt oss så att vi smidigt har kunnat korrekturläsa varandras texter.
\subsection{Diskussion}
\subsubsection{Resultat}
\subsubsection{Metod}
\subsection{Slutsatser}
\subsection{Referenser}
\vspace{-9mm}
\renewcommand{\refname}{}
\begin{thebibliography}{9}
\footnotesize
\bibitem{word_ursprung}
\url{http://ia801406.us.archive.org/21/items/A_History_of_the_Personal_Computer/eBook12.pdf}\\
 Sida 11-12. Hämtad 2015-04-20.

\bibitem{docs_launch}
\url{http://googlepress.blogspot.se/2006/06/google-announces-limited-test-on-google_06.html}\\
 Hämtad 2015-04-28.

\bibitem{upstartle}
\url{http://googleblog.blogspot.se/2006/03/writely-so.html}\\
 Hämtad 2015-04-28. 

\bibitem{donald_knuth}
\url{https://gcc.gnu.org/ml/java/1999-q2/msg00419.html}\\
 Hämtad 2015-04-20.

\bibitem{leslie_lamport}
\url{http://research.microsoft.com/en-us/um/people/lamport/pubs/pubs.html#latex}.\\
 Hämtad 2015-04-20.
 
\bibitem{overleaf}
\url{https://www.overleaf.com/}\\
 Hämtad 2015-04-28.

\bibitem{authorea}
\url{https://www.authorea.com/}\\
 Hämtad 2015-04-28.
 
\bibitem{latex_standard}
\url{ftp://ftp.dante.de/tex-archive/info/intro-scientific/scidoc.pdf}\\
Sida 2. Hämtad 2015-04-28.

\end{thebibliography}

