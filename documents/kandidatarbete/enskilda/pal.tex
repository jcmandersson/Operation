\section{Pål Kastman}
\subsection{Inledning}
I detta projekt så har vi valt att inledningsvis skriva dokumentationen i Googles ordbehandlingsprogram Google Docs för att 
i ett senare skede då det var dags att påbörja denna rapport gå över till att använda typsättningssystemet Latex. 

Vi valde att göra på detta sätt för att så snabbt som möjligt komma igång, då man bland annat i Google Docs kan se live 
vad andra skriver.

\subsubsection{Syfte}
Syftet med denna individuella del är att undersöka funktionaliteten i Latex och väga fördelar mot nackdelar.

\subsubsection{Frågeställning}
\begin{itemize}
\item Vad finns det för begränsningar i Latex
\item Kommer det att vara en fördel eller en nackdel för flera gruppmedlemmar att jobba samtidigt i samma delar av rapporten.
\end{itemize}

\subsubsection{Avgränsningar}
Denna rapport kommer att avgränsas för att endast jämföra Latex, Microsoft Office och Google Docs, detta för att 
inte behöva jämföra alla olika ordbenhandlingsprogram.

\subsection{Bakgrund}
Dokumentering är någonting som är viktigt att göra när man arbetar i projektform, dels för egen del utifall man behöver 
gå tillbaka och se vad som gjorts, men även för andras skull ifall man kanske får en ny medarbetare som ska integreras i 
projektet.

En fråga som man alltid behöver besvara är i vilket ordbehandlingsprogram dokumentationen skall skrivas. Det populäraste 
alternativet kan tänkas vara Microsoft Word vilket har funnits sedan 1983\cite{leslie_lamport}.

Ett annat alternativ som blir allt vanligare är Google Docs.

Ett tredje och inte lika vanligt alternativ är Latex. Detta är egentligen bara en vidareutveckling av typsättningssystemet 
Tex vilket skapades av den amerikanske matimatikern Donald Knuth 1978\cite{donald_knuth}. Leslie Lamport var mannen som vidareutvecklade detta 
språk i början av 80-talet, till det vi idag kallar Latex\cite{latex_ursprung}.




\subsection{Teori}
När Microsoft utvecklade Word på 80-talet så ville de naturligtvis att dåtidens datorer skulle klara av att ladda 
dokument utan att det skulle vara alltför krävande, därför konstruerade man dess filformat (.doc) binärt och gjorde konstruktionen 
väldigt komplex. Detta medförde att bara personer som var riktigt insatta i detta filformat klarade att ändra direkt i dessa filer.

Tex däremot är i grunden ett märkspråk liksom t.ex. HTML, där man istället ändrar direkt i koden genom att tilldela text olika taggar

\subsection{Metod}
Vi har till en början valt att under projektets gång arbeta i endast en fil för den gemensamma delen av kandidatprojektet. Vad det gäller 
de individuella delarna så har vi valt att lägga dessa i separata filer och sedan importera dessa till den gemensamma. Vi 
sparar alla dokumentfiler i samma repository på github som vi har källkoden.

Genom att göra på detta sätt så kan vi garantera att det inte kommer att uppstå några konflikter i de individuella delarna.
Det som vi inte vet, är huruvuda det kommer att uppstå problem då alla gruppmedlemmar skall skriva på den gemensamma delen.

\subsection{Resultat}
\subsection{Diskussion}
\subsubsection{Resultat}
\subsubsection{Metod}
\subsection{Slutsatser}
\subsection{Referenser}
\begin{thebibliography}{9}
\bibitem{leslie_lamport}
\url{http://ia801406.us.archive.org/21/items/A\_History\_of\_the\_Personal\_Computer/eBook12.pdf}.\\
 Hämtad 2015-04-20.
\bibitem{donald_knuth}
\url{https://gcc.gnu.org/ml/java/1999-q2/msg00419.html}.\\
 Hämtad 2015-04-20.
\bibitem{latex_ursprung}
\url{http://research.microsoft.com/en-us/um/people/lamport/pubs/pubs.html\#latex}.\\
 Hämtad 2015-04-20.
 
\end{thebibliography}