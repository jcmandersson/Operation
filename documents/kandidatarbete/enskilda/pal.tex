\section{Pål Kastman}
\subsection{Inledning}
I detta projekt har jag haft rollen som dokumentansvarig, en del av det arbete jag har utfört i denna roll har gått ut på att se till att dokumenten ser ordentliga ut, att ingenting är felstavat.

I denna individuella del av rapporten så undersöker jag hur det är att arbete med Latex i denna typ av projektform jämfört med andra orbehandlingsprogram.

\subsubsection{Syfte}
Syftet med denna individuella del är att undersöka funktionaliteten i Latex och väga fördelar mot nackdelar.

\subsubsection{Frågeställning}
\begin{itemize}
\item Uppmuntrar latex till en viss typ av samarbete jämfört med andra ordbehandlingsprogram?
\item Kommer det att vara en fördel eller en nackdel för flera gruppmedlemmar att jobba samtidigt i samma delar av rapporten?
\end{itemize}

\subsubsection{Avgränsningar}
Denna rapport kommer att avgränsas för att endast jämföra Latex, Microsoft Office och Google Docs, detta för att inte behöva jämföra alla olika ordbenhandlingsprogram.

\subsection{Bakgrund}
Dokumentering är någonting som är viktigt att göra när man arbetar i projektform, dels för egen del utifall man behöver gå tillbaka och se vad som gjorts, men även för andras skull ifall man kanske får en ny medarbetare som ska integreras i 
projektet. En fråga som man alltid behöver besvara är i vilket ordbehandlingsprogram dokumentationen skall skrivas. Det populäraste alternativet kan tänkas vara Microsoft Word, vilket har funnits sedan 1983 \cite{word_ursprung}.

Ett annat alternativ som blir allt vanligare är Google Docs, vilket är ett web-baserat ordbehandlingsprogram som lanserades av Google 2006 \cite{docs_launch} efter att man hade köpt upp företaget Upstartle \cite{upstartle}.

Ett tredje och inte lika vanligt alternativ är Latex, vilket inte är ett ordbehandlingsprogram utan istället ett märkspråk så som t.ex. HTML, där man istället för med ett grafiskt gränssnitt formaterar sin text,sätter sin text inom speciella taggar så att när koden senare kompileras får rätt utseende.

Latex är egentligen bara en uppsättning makron skrivna för språket Tex, vilket skapades av den amerikanske matimatikern Donald Knuth 1978 \cite{donald_knuth}. I början av 80-talet så vidareutvecklade Leslie Lamport Tex med hjälp av dess makrospråk till det som 
idag är Latex \cite{leslie_lamport}.

Vad det gäller min egen bakgrund i Latex så hade jag när detta projektet påbörjades endast använt det i ett tidigare projekt, under det projektet blev jag dock inte så insatt i Latex utan fyllde endast i material i de redan färdiga mallarna vi hade.

Under kursens gång har jag dock skrivit en laborationsrapport i Latex och därigenom skaffat mig lite mer erfarenhet.


\subsection{Teori}
\subsubsection{Microsoft Word och Google docs}
När Microsoft utvecklade Word på 80-talet så ville de naturligtvis att dåtidens datorer skulle klara av att ladda dokument utan att det skulle vara alltför prestandakrävande, därför konstruerade man dess filformat (.doc) binärt och gjorde konstruktionen väldigt komplex. Detta medförde att bara personer som var riktigt insatta i detta filformat klarade att ändra direkt i dessa filer.

Google docs däremot använder sig av "molnet" för att spara filer och man behöver därför aldrig oroa sig över säkerhetskopiering. Textformateringen i bägge dessa program görs främst genom att använda det grafiska gränssnittet, men kan även göras genom tangentbordskommandon. En funktion som finns i bägge dessa program är möjligheten att kommentera text.

\subsubsection{Latex}
Latex är i motsats till Word och Google docs inget ordbehandlingsprogram, utan istället ett märkspråk liksom HTML. När man i Word och Google docs ändrar textformateringen genom ett grafiskt gränssnitt så markerar man istället sin text med taggar, som senare när man kompilerar koden ger det önskade utseendet. 

%exempelkod vore bra här!
	
Detta gör att man som användare behöver bry sig mindre om utseendet av dokumentet och kan istället fokusera på innehållet. Eftersom Latex inte är något program i sig utan mer ett programspråk så behöver man något program att redigera koden i, detta kan göras i vilken textredigerare som helst, men det finns även program som kan kompilera koden åt användaren så att man kan se direkt vilka ändringar som görs. Det har på senare år dykt upp flera webbsidor där man kan skriva Latexdokument live, tillsammans med andra användare (ungefär som Google docs). Exempel på sådana sidor kan vara Overleaf \cite{overleaf}, eller Authorea \cite{authorea}. Latex har väldigt många inbyggda kommandon och det är väldigt lätt att t.ex. skriva formler vilket har gjort Latex till standard inom den vetenskapliga sektorn \cite{latex_standard}.


\subsection{Metod}
I detta projekt så har vi valt att inledningsvis skriva dokumentationen i Googles ordbehandlingsprogram Google Docs för att i ett senare skede, då det var dags att påbörja denna kandidatrapport gå över till att använda typsättningssystemet Latex. Vi valde att göra på detta sätt för att så snabbt som möjligt komma igång, då man bland annat i Google Docs kan se live vad andra skriver.

I denna rapport så skrev vi inledningsvis den gemensamma delen i samma fil, och de indivuella delarna i separata filer som vi sedan importerade till den gemensamma. I ett senare skede lade vi även ut vissa delar av den gemensamma rapporten i separata filer.

Genom att göra på detta sätt så kan vi garantera att det inte kommer uppstå några konflikter i de individuella delarna.

\subsection{Resultat}
I detta stycket presenteras resultatet, det är givetvis inte bara mina egna åsikter som är nedskrivna här utan jag har även lyssnat en del på vad de andra projektmedlemmarna har haft för åsikter och synpunkter.

\subsubsection{Dokumentstandard}
När vi i projektets inledning använde Google docs så märkte vi att det var väldigt svårt att hålla sig till någon sorts dokumentstandard där man bl.a. använder samma typsnitt på rubriker och text. Även om alla projektmedlemmar har försökt att hålla sig till den standard vi satt upp så blev det ändå ibland att någon del av dokumenten blev annorlunda när vi redigerade samtidigt i dessa. 

Detta fungerade bättre i Latex om än inte helt utan problem då det även här uppstod problem med att olika personer gjorde på olika sätt, dock kändes det som att det var enklare att reda ut dessa "missförstånd" än det var i Google docs då latex på ett helt annat sätt hjälper användaren till att hela tiden hålla samma standard i dokumenten.

\subsubsection{Revisionshantering}
Vi har under projektet försökt att hålla isär på alla olika revisioner av dokument så att man senare kan gå tillbaka och se vilka ändringar som har gjorts från en iteration till en annan, detta har varit svårt att upprätthålla med Google docs då dessa dokument hela tiden sparas kontinuerligt och det inte finns något sätt att se historiken av vad som ändrats i ett dokument. Detta har gjort att när vi ändrat i dokument, behövt spara undan en kopia av den föregående revisionen vilket inte alltid har gjorts. I kandidatrapporten däremot har vi använt oss utav revisionshanteringsprogrammet git och på så sätt lätt kunnat se vilka ändringar som gjorts. Detta har även hjälpt oss så att vi smidigt har kunnat korrekturläsa varandras texter. 

Användningen av git har dock inte helt varit en dans på rosor. Vi har ibland fått mergekonflikter \textbf{KÄLLA!} när vi laddat upp dokumenttext till github. Ett annat problem har varit att om en gruppmedlem lägger till en bild så måste denne komma ihåg att även ladda upp bilden till github, när man glömt detta så bidrar det till att andra gruppmedlemmar inte har kunnat kompilera sin kod efter att ha laddat ned senaste versionen. Just att se till att få filer till github är oftast inget problem när man använder webstorm eftersom detta program frågar om man vill lägga till filen till repositoryt.

\subsubsection{Kommentering}
En stor fördel med Google docs är att man kan lägga till kommenterar i texten så att andra kan se dessa, detta är en väldigt smidig funktion som gör att man lätt kan kommentera varandras texter. Sedan kan någon annan antingen lösa problemet eller svara på kommentaren. Denna funktionen finns som tidigare nämnts även i Word, men den stora skillnaden är att filerna först måste skickas mellan projektmedlemmarna först för att man ska kunna se dem. Det går i och för sig att lägga in kommentarer i Latex också men de blir ingenting man ser direkt i dokumentet utan man måste leta reda på dem i koden. Vi valde att lösa detta genom att man istället korrekturläser varandras texter på github och istället lägger in kommentarer där istället.


Detta är svårare att göra med Latex, vi har löst det genom att man istället kommenterar varandras texter i github, nackdelen med detta är att det givetvis inte sker live utan att man måste ladda upp texten först.

\subsection{Diskussion}
\subsubsection{Resultat}
Jag tycker helt klart att Latex har sina fördelar när man väl har lärt sig det, det är dock just inlärningskurvan som kan vara ett problem, speciellt om man är många som ska lära sig det. Som visat i resultat så kan man aldrig helt komma ifrån att dokument skrivs på olika sätt i alla fall till en början.

\subsubsection{Metod}
\subsection{Slutsatser}
\subsection{Referenser}
\vspace{-9mm}
\renewcommand{\refname}{}
\begin{thebibliography}{9}
\footnotesize
\bibitem{word_ursprung}
\url{http://ia801406.us.archive.org/21/items/A_History_of_the_Personal_Computer/eBook12.pdf}\\
 Sida 11-12. Hämtad 2015-04-20.

\bibitem{docs_launch}
\url{http://googlepress.blogspot.se/2006/06/google-announces-limited-test-on-google_06.html}\\
 Hämtad 2015-04-28.

\bibitem{upstartle}
\url{http://googleblog.blogspot.se/2006/03/writely-so.html}\\
 Hämtad 2015-04-28. 

\bibitem{donald_knuth}
\url{https://gcc.gnu.org/ml/java/1999-q2/msg00419.html}\\
 Hämtad 2015-04-20.

\bibitem{leslie_lamport}
\url{http://research.microsoft.com/en-us/um/people/lamport/pubs/pubs.html#latex}.\\
 Hämtad 2015-04-20.
 
\bibitem{overleaf}
\url{https://www.overleaf.com/}\\
 Hämtad 2015-04-28.

\bibitem{authorea}
\url{https://www.authorea.com/}\\
 Hämtad 2015-04-28.
 
\bibitem{latex_standard}
\url{ftp://ftp.dante.de/tex-archive/info/intro-scientific/scidoc.pdf}\\
Sida 2. Hämtad 2015-04-28.

\end{thebibliography}

