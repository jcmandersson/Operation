\documentclass{article}
\usepackage[utf8]{inputenc}
\usepackage[swedish]{babel}
\usepackage{tocloft}
\usepackage{etoolbox}
\usepackage{graphicx}
\graphicspath{ {images/} }
\usepackage{pdfpages}
\usepackage{float}
\usepackage{subcaption}
\usepackage{url}
\usepackage{listings}
%\usepackage{showframe}
%\usepackage[a4paper]{geometry}
\usepackage{fancyhdr}
\usepackage[parfill]{parskip}
\pagestyle{fancy}
\lhead{Grupp 7}

\rhead{\today}	
\cfoot{\thepage}

\renewcommand{\contentsname}{Innehåll}

\textheight = 600pt
\footskip = 30pt


\makeatletter
\@addtoreset{section}{part}
\makeatother

\renewcommand\cftpartpresnum{Del~}


\begin{document}
\pagenumbering{roman}

\begin{titlepage}
\begin{center}
  \textbf{\Huge Kandidatarbete}
\end{center}
\begin{center}
  {\Large Redaktör: Pål Kastman}
\end{center}
\begin{center}
  {\Large Datum: 2015-05-13}
\end{center}
\begin{center}
  {\Large \textbf{Version 1.0}}
\end{center}
\end{titlepage}
\newpage
\begin{flushleft}
  {\Large \textbf{Dokumenthistorik}}\\[0.5ex]
  \begin{center}
  \vspace{3mm}
    \begin{tabular}{ | l | l | p{5cm} | l |}
    \hline
    \textbf{Datum} & \textbf{Version} & \textbf{Utförda ändringar} & \textbf{Utförda av} \\ \hline
    2015-03-13 & 0.1 & Första utkast av gemensam och individuella rapporter & Pål Kastman \\ \hline
    2015-04-20 & 0.2 & Andra utkastet av gemensam och individuella rapporter & Alla \\ \hline
    2015-05-11 & 0.3 & Tredje utkastet av gemensam och individuella rapporter & Alla \\ \hline
	2015-05-13 & 1.0 & Fjärde utkastet av gemensam och individuella rapporter & Alla \\ \hline    
    \end{tabular}
  \end{center}
\end{flushleft}

\hfill

\begin{flushleft}
  {\Large \textbf{Projektidentitet}}\\[0.5ex]
  {\small} Detta dokument gäller för grupp 7  i kursen TDDD77 på Linköpings universitet
  \begin{center}
    \begin{tabular}{ | l | l | p{5cm} | l |}
    \hline
    \textbf{Namn} & \textbf{Ansvarsområde} & \textbf{E-post} \\ \hline
    Daniel Rapp & Teamledare & danth407@student.liu.se \\ \hline
    Daniel Falk & Analysansvarig & danfa519@student.liu.se \\ \hline
    Jonas Andersson & Arkitekt & jonan111@student.liu.se \\ \hline
    Albert Karlsson & Kvalitetssamordnare & albka735@student.liu.se \\ \hline
    Erik Malmberg & Testledare & erima694@student.liu.se \\ \hline
    Pål Kastman & Dokumentansvarig & palka285@student.liu.se \\ \hline
    Robin Andersson & Utvecklingsledare & roban563@student.liu.se \\ \hline

    \end{tabular}
\end{center}
\end{flushleft}

\begin{flushleft}
\textbf{Kund} \\ Region Östergötland \\
\end{flushleft}
\begin{flushleft}
\textbf{Kundkontakt} \\ 
Daniel Hall, daniel.hall@regionostergotland.se \\
Erik Sundvall, erik.sundvall@regionostergotland.se \\
Ingrid Hallander, ingrid.hallander@regionostergotland.se \\
\end{flushleft}
\begin{flushleft}
\textbf{Handledare} \\ Lena Buffoni, lena.buffoni@liu.se \\
\end{flushleft}
\begin{flushleft}
\textbf{Handledare} \\ Kristian Sandahl, kristian.sandahl@liu.se \\
\end{flushleft}

\newpage
\tableofcontents
\newpage
\pagenumbering{arabic}

\part{Gemensamma erfarenheter och diskussion}

\section{Inledning}
Detta avsnitt behandlar varför detta projekt utförs.

\subsection{Motivering}
Region Östergötland har idag ett system med handböcker som en sjuksköterska går igenom inför varje operation. I dessa handböcker finns bland annat förberedelse-uppgifter och plocklistor, som innehåller engångsmaterial. Handböckerna är idag inte interaktiva på något sätt, istället skrivs plocklistan och förberedelse\-uppgifterna ut och bockas av för hand. Plattformen med handböcker kan heller inte återanvändas på olika avdelningar på grund av licensproblem. 

Utöver detta system så finns ett annat separat system, som heter Kartoteket, för uppgifter om vilka artiklar som finns och var i lagret de ligger. Detta gör att personalen som ska förbereda inför operationer behöver gå in i två olika system. 

\subsection{Syfte}
Uppgiften som gruppen har fått är att skapa ett nytt system med handböcker som har interaktiva förberedelse- och plocklistor. Listorna ska uppdateras kontinuerligt när de bockas av så flera personer kan jobba på dem samtidigt. Plocklistan ska också innehålla uppgifter om var artiklarna ligger. Tanken är att personalen 
ska använda en iPad så de kan gå runt och plocka i lagret samtidigt som de bockar av plocklistan.

I mån av tid ska också extra funktionalitet implementeras. Till exempel sortera plocklistan med avseende på närmsta väg mellan artiklarna, lagersaldo och media i handböckerna.
Hela systemet ska ligga under den öppna källkodslicensen Apache 2.0 \cite{apache} så det kan användas och vidareutvecklas fritt av alla.    

\subsection{Frågeställning}
Rapporten ska besvara följande frågeställningar.
\begin{itemize}
\item Hur kan ett system för operationsförberedelser realiseras så arbetet blir lättare och mer effektivt?
\item Vilka strategier kan användas för effektiv utveckling i en grupp där kunskapsnivån varierar? 
\end{itemize}

\subsection{Avgränsningar}
Den här rapporten beskriver hur ett system för operationsförberedelser på universitetssjukhuset i Linköping kan realiseras. Andra sjukhus kan ha andra rutiner vilket får konsekvensen att systemet inte fungerar där.

Kunden ville också att ett lagersaldo skulle integreras i systemet och göra det möjligt att skanna av artiklar då dessa plockas. Vidare ville kunden även att systemet skulle ge möjlighet att välja mellan olika kliniker i regionen. Tiden kändes inte tillräcklig för att åstadkomma dessa implementeringar, därför klargjordes det, i ett tidigt skede, att grundläggande funktionalitet hade högre prioritet än lagersaldo och val av klinik. En kompromiss gjordes där dessa krav fick prioritet 2 vilket innebar att kraven var önskvärda och skulle implementeras i mån av tid, eller prioritet 3 där kraven sågs som en framtida utbyggnad.

Hur man ska jobba i projektgrupper med olika kunskapsnivåer tas fram genom undersökningar i projektgrupp 7 i kursen TDDD77 på Linköpings universitet. Resultaten av denna undersökning kan kanske därför inte appliceras i situationer som inte kan likställas med situationen i projektgruppen.

\section{Bakgrund}
\label{bakgrund}
Studenterna som studerar kursen TDDD77 fick i januari 2015 ett uppdrag att utföra ett kandidatarbete. Först fick gruppen rangordna flera projektuppdrag efter önskemål, för att sedan få ett av dessa uppdrag tilldelat sig. Projektgruppen fick då detta projekt som skickades in av Region Östergötland.

\subsection{Programspråk och bibliotek}
I detta projekt kommer följande programspråk och bibliotek att användas.

\subsubsection{Node.js}
Node.js är en plattform för att skapa applikationer till framförallt webbservrar \cite{nodejs}. Det finns en inbyggd pakethanterare vid namn npm som gör det enkelt att inkludera bibliotek i sina projekt. Det är därför väldigt enkelt att använda sig av ett bibliotek istället för att skriva all funktionalitet själv. Node.js är baserat på öppen källkod som är tillgänglig under en MIT-licens.

\subsubsection{KeystoneJS}
KeystoneJS är ett ''content management system'' som bland annat innehåller ett kraftfullt administrationsverktyg och en databashanterare \cite{keystonejs}.

\subsubsection{Socket.IO}
Socket.IO är en modul till Node.js och ett bibliotek för front-end \cite{socketio}. Socket.IO använder sig av websockets för att kommunicera mellan front-end och back-end. Med hjälp av Socket.IO så kan man skicka data från en klient till alla andra anslutna klienter och visa datan som skickades utan att någon sida behöver laddas om.

\subsubsection{MongoDB}
MongoDB är en dokumentbaserad databas som stödjer många olika plattformar \cite{mongodb}.

\subsubsection{Handlebars}
Handlebars är ett ''template language'' som är byggt på Mustache, vilket är ett annat ''template language''. Handlebars ger möjlighet att lägga in en del logik i HTML-kod \cite{handlebars}. Till exempel kan man göra loopar för att iterera över data och skapa HTML-kod för de olika objekten i en lista.  

\subsubsection{Less}
Less kan användas som vanlig CSS. Men det ger även möjlighet till att använda variabler och funktioner som sedan kompileras till CSS-kod \cite{less}. Till exempel kan en variabel för en färg användas istället för att direkt sätta färgen. Om färgen ska ändra så kan man då bara ändra på variabeln istället för att ändra på varje ställe.

\subsubsection{Bootstrap}
Bootstrap är det mest populära HTML, CSS och Javascript-ramverket för att utveckla responsiva och mobila webbapplikationer \cite{bootstrap}. 

\subsubsection{JQuery}
JQuery är ett funktionsrikt Javascriptsbibliotek. Det gör saker som manipulation och traversering av HTML-dokument, animationer och Ajax mycket enkelt \cite{jquery}.

\subsubsection{Wkhtmltopdf}
Wkhtmltopdf är ett program som tar en hemsida eller en HTML-fil och skapar en pdf av den \cite{wkhtmltopdf}. Programmet har många olika parametrar som kan ändras t.ex. mediatyp så pdf-filen kan se ut som en utskrift av hemsidan. För att kunna använda detta programmet med Node.js finns det också en Node.js-modul med samma namn som skickar kommandon till programmet.


\section{Teori}

För att kunna svara på frågeställningar så måste några begrepp i dem definieras.
\begin{itemize}
\item Effektivt definieras i detta fallet som att det tar kortare tid för operationsförberedelser och kräver mindre energi av personalen, till exempel att personalen inte ska behöva hålla lika mycket information i huvudet. 
\item Lättare definieras som att det går snabbare för nya personer att lära sig rutinerna och kunna utföra arbetet.
\end{itemize}

\subsection{Varför är effektivisering viktigt?}
Effektiviseringen av sjukvården och hur den ska ske är frågor som diskuterats flitigt de senaste åren. Region Östergötland har gjort tester för att kontrollera hur en operationsförberedelse kan effektiviseras. Ett av dessa tester gick ut på att undersöka hur lång tid det tar att plocka lagerartiklar då information om lagerplats finns lättillängligt gentemot hur lång tid det tar med nuvarande system. Resultatet av detta var att det tog 5 minuter för en person utan erfarenhet att plocka materialet då lagerplats fanns tillgängligt och 30 minuter för en sjuksköterska med lång erfarenhet med nuvarande system. Detta visar att det finns stora tidsvinster i att bygga ett system som underlättar operationsförberedelser.



 
\section{Metod}
%\subsubsection{Kundmöten}
%I inledningen hölls totalt fyra möten, det första var ett längre möte, där endast teamledaren, kundansvarige och arkitekten närvarade från projektgruppen. Detta möte hölls främst som en introduktion men var även ett tillfälle för kunden att visa upp sitt nuvarande system och få framföra ideér om hur system skulle kunna se ut och fungera istället. Sedan gjordes ett studiebesök hos kunden där alla gruppmedlemmar deltog. Man fick se hur en sjuksköterska förberedde inför en operation och även se deras nuvarande system i bruk. Under det tredje mötet så utformade man kravspecifikationen som man sedan spikade under det fjärde mötet.Efter varje iteration hölls ett möte på plats hos kunden för att diskutera hur arbetet fortskred, detta var även ett perfekt tillfälle att ta upp saker man var osäker kring, men var även bra för båda parter att föreslå ändringar i systemet. Mötet efter iteration två bestämdes att det skulle sättas upp en server där kunden skulle kunna testköra systemet, det bestämdes också att några från projektgruppen skulle närvara vid dessa tester. Detta gjordes två gånger där arkitekt och kundansvarige medverkade vid första tillfället, kvalitetsansvarige och dokumentansvarige medverkade vid det sista tillfället


Detta avsnitt beskriver kravinsamlingen och utveckling i projektet.
\subsection{Kravinsamling}

%Gruppen utgick ifrån ett projektdirektiv och började tänka över hur systemet skulle byggas. En introduktion till projektet gavs också vid ett första möte med kund. Under detta möte bestämdes också hur insamlingen av krav skulle gå till. Analysansvarig utsågs som ansvarig för detta. Denne skulle med hjälp av kunden arbeta fram en kravspecifikation som båda parter vara nöjda med. Detta gjordes genom flera möten och ett gemensamt dokument på Google drive där båda parter kunde gå in för att redigera och skriva kommentarer. Det kom fram ganska snabbt att ingen i gruppen hade koll på hur operationsförberedelser går till, vilket gjorde att utbildning inom detta krävdes. För att få mer insyn så gjordes ett studiebesök på universitetssjukhuset i Linköping.

Denna del beskriver hur kravframställningen gått till i detta projekt och hur den bidragit till systemets utveckling. Först beskrivs arbetet med att analysera kundens behov under förstudien. Sedan beskrivs hur kravspecifikationen arbetades fram och hur kraven valdes att representeras. Avslutningsvis beskrivs hur kraven validerades. I den enskilda utredningen ''D-Kravinsamlingsmetoder'' så beskrivs mer ingående vilka metoder och verktyg som använts för att ta fram krav. Fokus ligger där på prototyper och användartester.
%hur referera till enskild?
\subsubsection{Förstudie}
Den största delen utav analysarbetet skedde under förstudien. Här identifierades de olika intressenterna och en kravspecifikation utarbetades. Den första kontakten med projektet var en projektbeskrivning där kunden formulerade sina mål och visioner av projektet. Några viktiga krav gavs också såsom att prototypdesign skulle genomföras i samarbete med kunden. Ett första möte, av fyra, under förstudien gav sedan mer information och projektgruppen började sin kravinsamling. 

Det konstaterades att det fanns två olika intressenter att arbeta med. Dels sjuksköterskorna som är användare av systemet och dels CMIT, Centrum för medicinsk teknik och IT, som ansvarar för sjukhusets IT-miljöer. 

För att förstå användarnas behov genomfördes ett studiebesök där projektgruppen fick en visning av nuvarande system och hur det används. Detta var nödvändigt för att verkligen förstå vad som behövde göras och vad som kunde förbättras. Från CMIT:s sida genomfördes mer tekniska möten där teknikval diskuterades. Här var det viktigt att ta reda på vilka begränsningar som fanns och vilka val som passade våra och deras erfarenheter.

\subsubsection{Kravspecifikation i Google Docs}
Kravspecifikationen skrevs i Google Docs. Detta valdes eftersom kraven utarbetades tillsammans med kund. Ett gemensamt redigerbart dokument gav en möjlighet att arbeta på olika platser under kravframställningen vilket var effektivt. Kommenteringsfunktionen i Google Docs möjligheten att kommentera krav och föreslå förbättringar. Under interna möten och kundmöten var den gemensamma redigeringen också till nytta då kravformuleringar snabbt kunde genomföras.

\subsubsection{Kravrepresentation}
Kraven gavs en prioritetsordning för att kunna urskilja de mest väsentliga kraven. Detta kändes nödvändigt då arbetet begränsades av en tidsbudget. En prioritering av kraven gav utrymme för vidareutveckling i mån av tid.
\begin{itemize}
\item Krav med prioritet 1 var att betrakta som grundkrav som skulle genomföras för att projektet skulle ses som godkänt.
\item Krav med prioritet 2 var att betrakta som önskvärda och som skulle genomföras om grundkraven var uppfyllda.
\item Krav med prioritet 3 var krav som fångats upp men som skulle ses som framtida utbyggnad. 
\end{itemize}  
%Koppla till nån litteratur/standard.

%Strukturering av krav
Kraven numrerades för att lätt kunna refereras till under projektets gång. De delades också in i olika sektioner efter deras del i systemet. Sektionerna var plocklistor, handböcker, Kartotek, lagersystem. Två extra sektioner användes också för generella krav och för leveranser. Denna uppdelning kändes naturlig för detta projekt. 
%Man skulle också kunna dela in dem efter blablabla...

Kravspecifikationen skrevs med stöd från standarden IEEE 830 \cite{ieee830}. Enligt standarden ska ett krav vara korrekt, otvetydigt, färdigt, konsekvent, prioriterat, verifierbart, modifierbart och spårbart. Detta eftersträvades men det kan diskuteras om alla krav uppfyller detta. I slutändan var det ändå den gemensamma förståelsen för kravet tillsammans med kunden som accepterades. 

Enligt standarden uttrycktes kraven på ska-form. Ett exempel på ett krav från projektet är följande: \textit{''Plocklistor ska innehålla information om artikelns namn, förråd, sektion, hylla och fack''}.

\subsubsection{Kravvalidering}
Vid varje iterationsslut hölls ett möte med kund där systemet demonstrerades och kunden fick en chans att testa det. Iterationerna gjorde att projektgruppen snabbt kunde rätta till eventuella missförstånd. 

Vilka krav som var genomförda och vilka krav om skulle prioriteras till nästa iteration sågs över. Kravspecifikationen var på så sätt dynamisk och uppdaterades under projektets gång. En färgkodning användes för att markera vilka krav som var godkända, vilka som hade påbörjats och vilka som återstod.

\subsection{Utveckling}
Projektets utveckling har skett i fyra iterationer där den första var en förstudie och i de avslutande tre iterationerna har utvecklingen av produkten skett. Nedan beskrivs vilka hjälpmedel som har använts för att underlätta arbetet.

\subsubsection{Roller}
I detta projektet finns det sju olika roller. Dessa är teamledare, analysansvarig, arkitekt, kvalitetssamordnare, testledare, dokumentansvarig och utvecklingsledare. Vad dessa roller innebär kan ses nedan. 
\begin{itemize}
\item Teamledare: Har huvudansvar för att arbetet inom gruppen går bra och att de uppgifter som ska genomföras bli gjorda. 
\item Dokumentansvarig: Har ansvar för att alla dokument korrekturläses och ser bra ut. 
\item Kvalitetssamordnare: Innehar ansvaret för att en kvalitetsplan tas fram och följs. 
\item Testledare: Skriver testplan, testrapporter och har huvudansvar för att testning sker och görs på rätt sätt.
\item Utvecklingsledare: Tar fram verktyg för utveckling och leder utvecklingsarbetet.
\item Arkitekt: Skriver arkitektturbeskrivning och bestämmer hur applikationen ska utformas.
\item Analysansvarig: Har hand om kundkontakten och har huvudansvar för att en bra kravspecifikation tas fram.
\end{itemize}
Dessa roller delades ut efter gruppmedlemmarnas önskan. Då flera personer ville ha samma roll diskuterades de olika kandidaternas kvalifikationer, vilket följdes av ett gemensamt beslut från gruppen.

\subsubsection{SCRUM}
Projektet har utvecklats iterativt med en utvecklingsmetodik som påminner mycket om SCRUM. Aktiviteterna i projektet har visats på en gemensam SCRUM-board med hjälp av webbsidan Trello. 

\begin{figure}[H]
	\begin{center}
	\includegraphics[scale=0.5]{trello2.png}
	\caption{En Trelloboard}
	\label{fig:trello2}
	\end{center}
\end{figure}

Boarden hade kategorierna TODO, DOING och DONE enligt figur \ref{fig:trello2}. När en medlem valt en aktivitet så märktes aktiviteten med medlemmens namn och flyttades till den aktuella kategorin. 

Varje vecka har gruppen haft ett kort SCRUM-möte på 15 minuter. På mötet har varje gruppmedlem berättat vad som har gjorts den föregående veckan och vad som ska göras den kommande veckan. SCRUM-mötet har oftast utförts i samband med varje veckas handledarmöte.

\subsubsection{Alpha state cards}
Ett system med alpha state cards har använts för att gruppen skulle kunna ha koll på hur långt projektet har fortskridit. Korten har även använts till att identifiera aspekter av projektet som kan förbättras och som gruppen behöver arbeta mer med. De aspekter av projektet som kunnat förbättras har använts som mål för kommande iterationer. De olika korten har uppdaterats kontinuerligt under projektets gång.

\subsubsection{Sammarbete i gruppen}
Under utvecklingsfasen delades projektgruppen till en början upp i två grupper som fokuserade på front-end och back-end. Allt eftersom projektet fortgick så övergick de flesta av gruppmedlemmarna till att endast arbeta på front-end. Utvecklingen skedde nästan uteslutande i samma lokal.

%Under förstudien så diskuterades det hur utvecklingen skulle ske på bästa på sätt för gruppen. Förslagen som diskuterades var, utveckling i mindre grupper, separat arbete var för sig eller att alla gruppmedlemmar skulle sitta tillsammans och utveckla. Då alla i gruppen inte hade utvecklat någon webbaplikation tidigare så togs beslutet att under förstudien paralellt inleda en intern utbildning inom gruppen i form av så kallade kodstugor, dessa gick ut på att alla skulle sitta tillsammans och lära sig de olika verktygen så som t.ex. Javascript och CSS och att de mindre erfarna skulle ha en möjlighet att få hjälp av de mer erfarna.

\clearpage
\section{Resultat}
Denna del av dokumentet presenterar resultatet av vårt arbete.
Här kommer vi diskutera både hur systemet ser ut och används, samt hur
det är uppbyggt rent tekniskt.

\subsection{Översikt av systemet}
Det finns två huvuddelar i systemet. Handböckerna och kartoteket.

Handböckerna beskriver hur man förbereder olika typer av operationer.
Varje handbok har en lista med artiklar som behövs till operationen, en så kallad plocklista.
Artiklarna är kopplade till kartoteket, som bland annat innehåller information om var i förråden artiklarna finns placerade.

När en patient registreras skapas en instans av en handbok.
I denna instans kan man checka av en lista med artiklar som ska användas under operationen och även andra förberedelser.
En samordnare kan se en översikt på hur långt man har kommit med de förberedelserna för varje instans.
Se en översikt i figur \ref{fig:overview}.

\begin{figure}[htbp]
  \centering
  \includegraphics[width=0.9\textwidth]{images/overview.png}
  \caption{Översikt av systemet}
  \label{fig:overview}
\end{figure}

\subsubsection{Server och klient}
Applikationen består av en server som distribuerar en hemsida till flera klienter.
Servern kan köras i en Windowsmiljö.
Hemsidan är responsiv och fungerar på surfplattor och datorer i olika format.

\subsubsection{Kartoteket}
I kartoteket finns information om alla artiklar som Region Östergötland har i förråden.
Här finns bland annat information om var artiklarna är placerade samt information relaterade till inköp av artiklar.

Läs mer om kartoteket i "Kartoteket" nedan.

\subsection{Tekniker}
Vi börjar med att beskriva tekniken bakom systemet.

\subsubsection{Översikt}
Programmet är uppdelat i två delar, en serverdel och en klientdel.
Serverdelen består av databaskopplingar som kopplas ihop och distribueras ut genom hemsidor till klienterna.

\begin{figure}
  \centering
  \includegraphics[width=0.9\textwidth]{images/techoverview.png}
  \caption{En översikt över tekniken}
  \label{fig:techoverview}
\end{figure}

I figur \ref{fig:techoverview} kan man se en översikt över de mest betydelsefulla tekniker och bibliotek som används för att bygga upp programmet.

\subsubsection{Back-end}
Koden till servern har skrivits helt i javascript.
Dels för att underlätta inlärningskurvan genom att använda samma språk som på klienten och dels för att realtidskommunikation mellan klienterna underlättas.
Grunden till programmet är node.js vilket är en plattform för att utveckla självständiga program i javascript med inbyggd pakethanterare.
Pakethanteraren gör det enkelt att installera externa bibliotek.

Det största och mest betydelsefulla ramverket för detta projekt är KeystoneJS, ett CMS-ramverk till node.js.
Följande är några punkter KeystoneJS underlättar:
\begin{itemize}
  \item Hjälper till att abstrahera systemet. Se avsnittet om MVC nedan.
  \item Skapar automatiskt en administreringssida för varje databasmodell.
    Huvuddelen av administeringen har dock bytts ut då den automatgenererade kan vara något begränsad.
  \item Har ett inbyggt användarsystem som är lätt att modifiera och byta ut.
  \item Sköter all kommunikation över http-protokollet till klienterna, dvs. gör hemsidan åtkomlig.
  \item Har inbyggt stöd för templatespråk som exempelvis Handlebars\footnote{TODO: bättre referens. http://handlebarsjs.com/ (2015-04-18)} och Less\footnote{TODO: Bättre referens. http://lesscss.org/ (2015-04-18)}.
\end{itemize}

För realtidskommunikation används ett programmeringsinterface vid namn socket.io\footnote{TODO: bättre referens. http://socket.io/ (2015-04-18)} användas.
Det är väldigt smidigt eftersom det väljer automatiskt hur datan ska skickas beroende på vilken webbläsare som används och vad den stödjer.
Socket.io är event-baserat vilket betyder att man skapar events på antingen klient eller serversida som man sedan kan trigga från motsatt sida.
Vanliga javascript-object kan skickas tillsammans med eventen.

\subsubsection{Front-end}
På klientsidan används bootstrap\footnote{TODO: Bättre referens.} och jQuery\footnote{TODO: Bättre referens.}.
Bootstrap har många färdiga CSS-klasser så man behöver inte skriva lika mycket CSS själv.
De klasser som finns är lätta att använda för att skapa responsiva hemsidor.
All kod är dessutom testad för att fungera på olika webbläsare vilket kan vara krångligt att lösa om man skriver all CSS från grunden.
JQuery används för att lättare hämta ut ett element på hemsidan och ändra data i det.
Det finns även många bra jQuery-bibliotek att hämta som gör att man slipper “uppfinna hjulet” i många fall.

För att ytterligare design av hemsidan används LESS\footnote{TODO: Referens.} att användas.
Det är en påbyggnad till CSS som kompilerar till vanliga CSS-filer.
Fördelen med LESS är bland annat att man kan använda variabler och enkla funktioner.
Exempelvis kan variablerna användas för att spara de olika färgerna på hemsidan för att enkelt kunna byta ut dem.


\subsubsection{Struktur}
En nackdel med jQuery gentemot andra bibliotek som exempelvis angular\footnote{TODO: Referens.} är att det lätt blir spaghettikod\footnote{TODO: Måste defineras?}.
För att strukturera så bra som möjligt är det viktigt att koden uppdelad.
För att uppnå detta så kommer varje enskild sida ha minst en skriptfil, en cssfil och en htmlfil.
På vissa sidor är det lätt att det blir stora skript och då bör man dela upp skriptfilen.
Den bästa lösningen är att abstrahera delar av skriptet till egna biblioteksfiler som man kan ladda in och använda.

\subsubsection{Säkerhetskopiering}
Ett system för automatisk säkerhetskopiering i form av en PDF-fil för varje handbok kommer att finnas.
Varje gång en handbok ändras kommer PDF-filen att uppdateras för att matcha handboken.
PDF-filen ska vara utformad så att operationerna kan förberedas utan hjälp från hemsidan.

\subsection{Mekanismer}
Här beskrivs syftet och funktionalitet hos olika delar av produkten.

\subsubsection{Översikt}
Som tidigare nämnts finns en översikt över alla operationsförberedelser.
Denna sida visar alla operationsförberedelser och hur långt är fortskridna.
Här används socket.io för att hela tiden hålla information uppdaterad.

\begin{figure}
  \centering
  \includegraphics[width=0.6\textwidth]{images/site/overview.png}
  \caption{Bild på översiktsvyn}
  \label{fig:siteoverview}
\end{figure}

I figur \ref{fig:siteoverview} kan man se hur förberedelserna är kategoriserade beroende på vilken kirurgisk specialitet handboken tillhör.
Ibland är det olika samordnare beronde på specialitet och det är då enkelt för en samordnare att hitta de operationerna som personen är ansvarig över.

\subsubsection{Handbok}
En handbok innehåller information om en operationsförberedelse.

\begin{figure}
  \centering
  \includegraphics[width=0.6\textwidth]{images/site/handbok.png}
  \caption{En handbok}
  \label{fig:handbok}
\end{figure}

All information i en handbok är uppdelad i olika rubriker. Dessa rubriker kan i sin tur vara uppdelade i olika processer.
I figur \ref{fig:handbok} ser man dels de olika processerna (Preoparea, anestesi, allmänt och operation) samt rubrikerna som hör till processen Anestesi (Övrigt, Premedicinering, blodrekvisition och  antibiotika).
Man kan även se att handböckerna har stöd för bilder.

\subsubsection{Sökfunktion}
Ett krav är att det ska vara lätt att hitta en handbok.
Därför kan man söka både på operationens namn men även på alternativa sökord (taggar), dessa är bra att ha då de medicinska termerna ibland kan vara svåra att komma ihåg.

\begin{figure}
  \centering
  \includegraphics[width=0.6\textwidth]{images/site/search}
  \caption{Sökfunktionen}
  \label{fig:search}
\end{figure}

I figur \ref{fig:search} kan man se sökresultaten där namnet på operationen står i större storlek, och sökorden kursivt i mindre storlek.

\subsubsection{Lista med handböcker}
Applikationen innehåller en enkel lista med alla handböcker.
Den går att sortera på valfri kolumn och kan även gruppera beroende på vilken specialitet handboken tillhör.
Se figur \ref{fig:list}

\begin{figure}
  \centering
  \includegraphics[width=0.6\textwidth]{images/site/list}
  \caption{Lista med handböcker}
  \label{fig:list}
\end{figure}

\subsubsection{Administrering}
I administreringsvyn kan man utöver att redigera all information även sortera processer och rubriker genom att dra och släppa dem.
Redigeringen under rubrikerna använder sig av en wysiwyg-editor.
Innan en redigerad eller ny handbok publiceras måste den granskas av en annan person.

\subsubsection{Operationsförberedelse}
En operationsförberedelse är nästan likadan som en handbok.
Det som skiljer är att vissa av rubrikerna kan gå att kryssa av när de är klara.

\begin{figure}
  \centering
  \includegraphics[width=0.6\textwidth]{images/site/op}
  \caption{Förberedelsevyn}
  \label{fig:op}
\end{figure}

I figur \ref{fig:op} kan man se att rubrikerna går att kryssa av när de är klara. Jämför med figur \ref{fig:handbok} där rubrikerna inte går att kryssa av.

\subsubsection{Plocklista}
En operationsförberedelse har även en plocklista med artiklar som behövs till operationen.

\begin{figure}
  \centering
  \includegraphics[width=0.6\textwidth]{images/site/plocklista}
  \caption{Plocklista}
  \label{fig:plocklista}
\end{figure}

I figur \ref{fig:plocklista} kan man även se att det finns stöd för att lägga en kommentar på en artikel.
Det är vanligt att en artikel är slut eller utbytt och man kan då lägga en kommentar på varför man inte kunde hämta den artikeln och även hur man har löst det istället.

\subsubsection{Kartoteket}
Kartoteket diskuteras i djup detalj nedan.

\subsection{Utvecklingen}
Här nedan beskrivs resultatet av hur utvecklingen inom gruppen har fungerat, och huruvida de olika hjälpmedlen har hjälpt eller stjälpt.

\subsubsection{SCRUM}
Utvecklingen i SCRUM-form har fungerat bra, dock har utvecklingen som beskriven i metod inte varit i rent SCRUM-format utan har modifierats för att bättre passa gruppen. Ett möte med handledare har hållits en gång i veckan där handledaren har haft möjlighet att ta upp saker som denne tyckt har behövts, sedan har gruppen hållit ett kort SCRUM-möte där man fått förklara vad man gjort förra veckan och vad man planerat att göra kommande vecka. Under mötets gång har man noterat om det är någonting som har behövt diskuteras vidare och detta har då noterats, varpå gruppen har gått igenom dessa frågor efteråt. Detta har fungerat väldigt bra då man på detta sättet inte har fått några större avbrott och alla hela tiden kunna hålla fokus. Efter SCRUM-mötena så har ordet varit fritt och alla har haft möjligheten att komma med synpunkter eller åsikter över någonting.

\subsubsection{Alpha state cards}
Korten har kontinuerligt uppdaterats av teamledare, dock har de kanske inte använts lika flitigt av gruppen som tanken var att de skulle?

\subsubsection{Kundmöten}
I inledningen hölls totalt fyra möten, det första var ett längre möte, där endast teamledaren, kundansvarige och arkitekten närvarade från projektgruppen. Detta möte hölls främst som en introduktion men var även ett tillfälle för kunden att visa upp sitt nuvarande system och få framföra ideér om hur system skulle kunna se ut och fungera istället. Sedan gjordes ett studiebesök hos kunden där alla gruppmedlemmar deltog. Man fick se hur en sjuksköterska förberedde inför en operation och även se deras nuvarande system i bruk. Under det tredje mötet så utformade man kravspecifikationen som man sedan spikade under det fjärde mötet.

Efter varje iteration hölls ett möte på plats hos kunden för att diskutera hur arbetet fortskred, detta var även ett perfekt tillfälle att ta upp saker man var osäker kring, men var även bra för båda parter att föreslå ändringar i systemet. Mötet efter iteration två bestämdes att det skulle sättas upp en server där kunden skulle kunna testköra systemet, det bestämdes också att några från projektgruppen skulle närvara vid dessa tester. Detta gjordes två gånger där arkitekt och kundansvarige medverkade vid första tillfället, kvalitetsansvarige och dokumentansvarige medverkade vid det sista tillfället

\subsubsection{Samarbete i gruppen}
Samarbetet i gruppen har fungerat väldigt bra, kodstugorna som gruppen höll under förstudien var alla nöjda med, de som var mindre erfarna med webbprogrammering kände att de fick all den hjälp de behövde när de körde fast i sitt arbete och därför kändes det naturligt för hela gruppen att fortsätta med dessa även under utvecklingsfasen.

%\subsection{Gruppens gemensamma erfarenheter}
%\subsection{Översikt över de inviduella utredningarna}

\clearpage

\section{Diskussion}
Här diskuteras metod och resultat i rapporten.
\subsection{Resultat}
Under iteration 3 utförde kunden användartester av systemet vid två tillfällen. En utvärdering av dessa tester bifogas i Appendix A.
%kanske kan använda nån fancy latex-referens till appendix egentligen.

\subsubsection{Back-end}
Back-end valdes att skrivas i Javascript dels för att minska inlärningskurvan eftersom det är samma språk som används på front-end och dels för att realtidskommunikation mellan klienterna underlättas. Node.js valdes som plattform på back-end eftersom den har en väldigt enkel pakethanterare. Detta underlättar när flera personer arbetar i ett projekt eftersom alla paket installeras automatiskt.

KeystoneJS underlättade utvecklingen väldigt mycket till en början. Några punkter som beskriver vad KeystoneJS hjälper till med följer:

  \begin{itemize}
    \item Skapar automatiskt en administreringssida för varje databasmodell.
      Huvuddelen av administeringen har dock bytts ut då den automatgenererade är något begränsad.
    \item Har ett inbyggt användarsystem som är lätt att modifiera och byta ut.
    \item Sköter all kommunikation över http-protokollet till klienterna, dvs. gör hemsidan åtkomlig.
    \item Har inbyggt stöd för Handlebars och Less.
  \end{itemize}
  
\subsubsection{Front-end}
Anledningen till att Bootstrap används på front-end är för att Bootstrap har många färdiga CSS-klasser, vilket leder till att gruppmedlemmarna inte behöver skriva lika mycket CSS själva.
De klasser som finns är lätta att använda för att skapa responsiva hemsidor.
All kod är dessutom testad för att fungera på olika webbläsare vilket kan vara krångligt att lösa om man skriver all CSS från grunden.

JQuery används för att lättare hämta ut ett element på hemsidan och ändra data i det.
Det finns även många bra jQuery-bibliotek som gör att man slipper ''uppfinna hjulet'' i många fall. En nackdel med jQuery gentemot andra bibliotek som exempelvis AngularJS \cite{angular} är att det lätt blir rörig kod.

Fördelen med Less är bland annat att man kan använda variabler och enkla funktioner i stilmallarna.
Exempelvis kan variablerna användas för att spara de olika färgerna på hemsidan för att enkelt kunna byta ut dem.

\subsubsection{Säkerhetskopiering och reservsystem}
Det som kan göras för att få säkerhetskopieringen bättre är att implementera att en ny pdf-kopia skapas så fort en operationsförberedelse har uppdaterats.

Det krävs mycket arbete för att lägga in alla handböcker, vilket gör att det är viktigt att databasen säkerhetskopieras med jämna mellanrum. Men det tar en del datorkraft att göra en säkerhetskopiering. Därför är det bra att det går att ställa in tidsintervallet som bestämmer hur ofta säkerhetskopiering av databasen ska ske.

\subsubsection{Översiktssida}
Överssiktssidan är tänkt att den ska visas på en monitor och inte kräva någon interaktion för att se resultat för olika operationsförberedelser. Det är därför viktigt att sidan använder realtidskommunikation och att allting uppdateras automatiskt. 

\subsubsection{Handbok}
För att göra övergången mellan det nya systemet och det gamla så gavs handböckerna ett utseende som liknar det gamla systemet till stor del. Det nya systemet har förbättrats genom att man har större förmåga att formatera innehållet efter behov. Man kan nu zooma i bilder, vilket är en förbättring mot föregående system.
Den största förändringen ligger i plocklistan som har en koppling direkt till Kartoteket.

Sökfunktionen ger möjligheten att snabbt hitta en handbok. Den placerades i sidhuvudet för att det alltid ska gå snabbt att ta sig till en handbok. Den har utformats efter kundens krav med möjlighet att använda alternativa sökord.
Sökfunktionen kompletterades med möjligheten att lista alla handböcker. Detta var också ett krav från kunden. 

Handbokens tillstånd har utformats för att tillfredsställa kraven. Denna utformning kunde dock ha gjorts på andra sätt. I kundens interna utvärdering (se Appendix A) nämns att tillstånden utkast och granskning upplevs som något förvirrande. Under testningen sågs granskningen som överflödig.
\subsubsection{Redigering}
I redigeringen av handböcker fokuserades det på att göra gränssnittet så likt en publicerad handbok som möjligt. Redigeringen var den svåraste delen att göra lättförståelig och användarvänlig. Det fanns flera designideér som skrotades under projektets gång.

Bland annat var det tvunget att göras en avvägning mellan flexibilitet och användar\-vänlighet.
Till exempel så förekommer ofta samma information under flera processteg. Man skulle då kunna tänka sig en lösning där denna information kopplas till flera processteg. Detta valdes dock bort eftersom projektgruppen ansåg att denna lösning skapar för stor komplexitet.

\subsubsection{Plocklista}
Plocklistan har utformats efter kundens behov. Den har under projektets gång uppdaterats många gånger efter feedback från kund. Efter användartester framkom att andra typer av artiklar, som inte finns i Kartoteket, med fördel också kunde ligga i en plocklista. Ett sätt att lösa detta på är att lägga in dessa artiklar i Kartoteket eller att skapa en ny databas för dessa med en separat plocklista.

\subsubsection{Roller}
Rollerna i gruppen följdes till viss del. Vissa rollers arbete distribueras mer eller mindre över hela gruppen. En sådan roll var utvecklingsledare. Utvecklingsledaren ledde inte utvecklingen utan gruppen skötte det tillsammans, vilket fungerade bra. Detta kanske inte fungerar lika bra i en grupp med sämre kommunikation. För att få rollen att fungera bättre i sådana situationer så kunde den definierats mer. Exempeluppgifter på vad utvecklingsledaren skulle göra hade gjort rollen tydligare. 

Kvalitetssamordnaren gjorde inte mycket mer kvalitetsarbete än att utforma kvalitetsplanen. Detta berodde främst på att funktionalitet prioriterades framför kvalitet i projektet och att det kvalitetsarbete som gjordes delades ut på alla gruppmedlemmar. Dokumentansvarig hade också en roll där det mesta av arbetet delades ut på alla gruppmedlemmar. 

Anledningen till att arbetet distribuerades var för att både kvalitetssamordnarens och dokumentansvarigs arbete hade blivit mycket enformigt med att granska dokument och kod. Detta fungerade också ganska bra för projektgruppen. Olika personer är ju dock olika bra på att granska kod och dokument, vilket leder till en viss ojämnhet i arbetet. I systemet som användes så var det bestämt vem som granskade vems kod och dokument. Ett roterande schema hade kunnat göra så att det blev en jämnare nivå. Detta hade dock gjort hela processen mycket krångligare. 

Att arbetet distribuerades ledde till att dessa roller kom i skymundan. Detta ger en svår avvägning. Om en roll blir för liten så finns risken att den inte tagits på allvar och att låta en person göra allt jobb inom rollen kan få konsekvenser om inte denne är intresserad av jobbet som rollen innebär. I detta projektet är det svårt att komma undan detta problem då sju stycken olika personer sammanförs och ska fördela sju roller. I större projekt och projekt hos företag är detta ett mindre problem då det finns fler personer att välja på. Ett företag anställer inte heller till exempel en kvalitetssamordnare som inte är intresserad av kvalité.

Den person med mest erfarenhet av webbutveckling blev arkitekt. Detta var ett mycket bra val då detta gjorde att en stabil grund kunde tagits fram snabbt för att arbetas vidare på. Om någon med mindre erfarenhet fått denna roll hade det tagit mycket längre tid att komma igång eftersom denne person då hade fått göra mycket mer efterforskning och hade haft svårt att tyda resultaten. 

Personerna som hade minst erfarenhet av webbutveckling fick rollerna kvalitetssamordnare, dokumentansvarig och analysansvarig. Detta var bra då dessa roller inte kräver så mycket förkunskaper. 


\subsubsection{Kodstugor}
Att arbetet oftast skedde på samma ställe gjorde att gruppmedlemmarna enkelt kunde fråga varandra om de fick problem vilket kunde spara dem många timmar av felsökning. Eftersom det var varierad kunskapsnivå på deltagarna i projektet så var det också lättare för alla att komma igång och jobba när de snabbt och enkelt kunde fråga om saker som de inte förstod. Det behövdes färre möten eftersom gruppmedlemmarna alltid visste vem som gjorde vad och vad som behövdes göras.

Att sitta på samma ställe och jobba skulle inte fungera lika bra i större grupper då det skulle bli alldeles för många kommunikationsvägar. 

\subsection{Metod}
Här diskuteras vad som fungerade bra och vad som kunde göras bättre med metoden.
\subsubsection{Kravinsamling}
Metoden som användes för kravframställning fokuserade på att samla in så mycket krav som möjligt genom möten, intervjuer och observationer. Att arbeta fram kraven tillsammans med kund kändes nödvändigt för att få en fullständig bild. 

En sak som inte prioriterades men som kanske hade kunnat hjälpa arbetet skulle varit att lägga större fokus på olika roller. Slutsystemet har rollerna administratör och icke-administratör. En avvägning gjordes där dessa två enkla roller valdes. Fördelen var att det kunde läggas fokus på att testa andra delar av systemet med högre prioritet snabbare. Nackdelen var att dessa funktionaliteter inte hann testas och under förstudien ledde till en viss förvirring. Vid användartesterna upplevdes inte detta som att det saknades vilket kan ses som att projektgruppen gjorde en bra avvägning. 

Att använda use-cases eller user-stories diskuterades också under förstudien. Detta bortprioriterades då de ansågs överflödiga. Denna avvägning ledde till att det var lite svårare att kommunicera kraven. 

Studiebesöket under förstudien var bra för att få en inblick i arbetet. Det hade dock behövts mer tid för observation. Vid studiebesöket var hela gruppen med och det blev lite stressigt och mycket folk. Därför kunde det varit bättre att ha fler studiebesök med färre gruppmedlemmar.

\subsection{Ur ett större perspektiv}
Applikationen som projektgruppen har utvecklat kan leda till stora fördelar för sjukhuset. Patientsäkerheten ökar då flödet för operationsförberedelser lätt kan följas. Den ökar också genom att det blir mindre felplockning av material eftersom materialets position och artikelnummer nu står i handböckerna. Att dessa automatiskt uppdateras i alla handböcker när en artikel ändras eller flyttas leder också till mindre felplockningar. 

Genom att positionen för material syns i handböckerna så går det mycket snabbare och enklare att plocka allt material. Detta leder till att personal känner sig mindre stressade och sjukhuset sparar pengar, vilket i sin tur kan leda till mer resurser inom andra delar av vården.

\newpage
\section{Slutsatser}
Den första frågeställningen
\begin{quote}
Hur kan ett system för operationsförberedelser realiseras så arbetet blir lättare och mer effektivt?
\end{quote}
har besvarats i tidigare kapitel. Men de främsta funktionaliteterna som finns är att materialets position syns på ett tydligt sätt och att denna informationen automatiskt uppdateras i alla handböcker då den ändras i Kartoteket. Att systemet är både lättare och mer effektivt än det gamla systemet har konfirmerats genom användartester utförda av kunden (se Appendix A).

%Vi har lyckats göra ett system som innehåller handböcker för operationer som kan användas för operationsförberedelser. Handböckerna innehåller bland annat en plocklista av engångsartiklar som behövs till olika operationer. Vid engångsartiklarna finns data om var engångsartikeln kan hämtas. Engångsartiklarna blir också sorterade efter var de finns så att man inte kommer att behöva springa fram och tillbaka mellan olika förråd och hyllor för att hämta artiklarna. Eftersom plocklistan uppdateras i realtid så kan också flera sjuksköterskor hämta artiklar samtidigt och checka av de artiklar som de plockat, vilket gör att det inte kommer att bli några konflikter med att flera sjuksköterskor plockar samma artikel. Detta system effektiviserar arbetet vid operationsförberedelser eftersom sjuksköterskor inte längre kommer behöva gå till en dator och söka på artikeln de ska ha för att få reda på i vilken hylla den finns. Syftet med detta projekt är därav avklarat.
Svaret på den andra frågeställningen
\begin{quote}
Vilka strategier kan användas för effektiv utveckling i en grupp där kunskapsnivån varierar? 
\end{quote}
är primärt att dela upp arbetet i olika roller, samt att arbeta i samma lokal för att underlätta kommunikationen inom gruppen.

\section{Fortsatt arbete}
Här följer en sammanfattning om fortsatt arbete med systemet. Restlistan behandlar krav som inte implementerats eller som borde vidareutvecklas. Även förslag på vidareutveckling som inte tas upp av kravspecifikationen tas upp. Dessa förslag är sådant som kommit fram under utvecklingens gång.

\subsection{Restlista}
Restlista för kunds räkning. 
\subsubsection{Krav som bör vidareutvecklas}
Följande krav är markerade som gula i kravspecifikationen vilket betyder att de bör vidareutvecklas för att ses som klara.
%%Fyll på gula krav här efter iteration 3 mötet
\begin{description}
\item[Krav nr 2] Systemet ska kunna köras på mobil (iphone).

Systemet ska i första hand köras på dator och surfplatta. Därför har detta krav lämnats för eventuell vidareutveckling.

\item[Krav nr 7] Systemet ska ha en funktion för att kunna göra backup av databaserna.

Detta krav är i princip uppfyllt men behöver funktionen behöver konfigureras för att fungera hos kund. 
%Bra om nån med koll fyller på
\item[Krav nr 10] Avcheckning av plocklistor och förberedelselistor ska under pågående operationsförberedelse uppdateras automatiskt/kontinuerligt och tillåta växling mellan inmatningsenheter.

Detta krav ligger kvar för vidareutveckling då en bugg upptäcktes vid slutmötet i iteration 3. Avcheckningen uppdaterades inte på surfplatta.%Denna kanske vi kan fixa innan sista inlämning?
\item[Krav nr 14] Separata plocklistor/påfyllnadslistor ska kunna skapas för t.ex. operationssalar.

Rent tekniskt kan man skapa en handbok för en specifik sal som tillfredställer detta krav. Det finns dock ingen speciell funktionalitet för detta.
\item[Krav nr 29] Administratörer ska kunna lägga till video och ljud i de olika underrubrikerna.

Det går att länka till externa video/ljudfiler. De hanteras dock inte inom systemet. %Formulera om iaf nån kan
\item[Krav nr 38] Behörighetsstyrning ska finnas för att kunna tilldela användare rättigheter att administrera innehållet i Kartoteket.

Det finns i dagsläget bara en behörighetsnivå som gäller för hela systemet. En vidareutveckling vore att ge användare mer specifika tillstånd som detta.
\item[Krav nr 42] När en artikel tas bort från Kartoteket ska det synas att den utgått i handboken.

Detta krav är inte uppfyllt då en bugg upptäcktes vid mötet i iteration 3. %Denna kanske vi kan fixa innan sista inlämning?
\end{description}

\subsubsection{Ej påbörjade krav}
Följande krav är markerade som röda i kravspecifikationen vilket betyder att de inte är påbörjade.
%%Fyll på röda krav här efter iteration 3 mötet

\begin{description}
\item[Krav nr 5]  I systemet ska användare kunna välja instans (operationsklinik med tillhörande handböcker och Kartotek) av applikationen. 

Detta går inte att göra i dagsläget. %Fyll på någon med koll om hur det skulle kunna lösas.
\item[Krav nr 17] Det ska finnas en importfunktion där handböcker kan läsas in från ifylld excelmall.

Om detta krav ska vidareutvecklas måste en tydlig excelmall tas fram.
\item[Krav nr 22] Påminnelsefunktion (t.ex. mailutskick) när handbok är äldre än ett angivet datum ska finnas.

Denna funktionalitet finns inte. %Fyll på någon med koll om hur det skulle kunna lösas.
\item[Krav nr 30] Multimedia ska hanteras inom programmet och inte i någon extern applikation.

Detta krav kräver mer tanke om hur det ska hanteras. Det är inte säkert att vinsten blir så stor vid vidareutveckling.%tankar?
\item[Krav nr 43] Kartoteket ska kunna skilja på vilket sjukhus och klinik som artikeln tillhör. (Viktigt då artiklar av samma sort finns men tillhör olika kliniker). 

Detta fungerar inte i dagsläget. %Fyll på någon med koll om hur det skulle kunna lösas.
\end{description}


Alla krav som behandlar lagersystemet finns kvar. För att utveckla ett lagersystem bör en ny förstudie genomföras för att ta fram tydligare krav. Ett lagersystem med skanning av artiklar skulle bidra till att plockningen blir mer säker. Det skulle också bidra till effektivare inventering och översikt av lagret. Ett saldo för varje artikel skulle kunna användas som beställningsunderlag. Kraven rörande lagret som samlats in under förstudien är följande:
\begin{description}
\item[Krav nr 46] Ska finnas ett saldo för varje artikel i förrådet.
\item[Krav nr 47] Ska gå att skanna av artiklar (streckkod) som plockas i en specifik plocklista.
\item[Krav nr 48] Ska gå att lagra in en artikel till förrådet som inte blivit använd genom att skanna in den.
\item[Krav nr 49] Saldot ska kunna användas som underlag till beställning.
\item[Krav nr 50] Beställningsunderlaget ska ha ett sådant format att det går att importera i Agresso.
\item[Krav nr 51] En inventeringsfunktion ska finnas.
\end{description}

\subsubsection{Övriga förslag}
Här följer övriga förslag på vidareutveckling.
%%Fyll på här om ni kommer på något!

\begin{description}
\item[Artikelsök] Vid sökning på artikel vid redigering av handbok kan det vara bra om man sorterar resultaten så att de artiklar som tillhör samma enhet som handboken visas först.
\item[Avancerad sökfunktion] En mer avancerad sökfunktion med * och sökning på flera ord som inte behöver vara direkt efter varandra.
\item[Förvalda kommentarer] Under användartester observerades att samma kommentarer skrivs ofta så det kan vara bra med en lista på vanliga kommentarer som man kan välja ifrån.
\item[Roller och rättigheter] Olika roller bör identifieras och en funktion för att tilldela användare olika rättigheter bör implementeras. I dagsläget kan en användare tilldelas rollen administratör. En användare utan denna behörighet har samma rättigheter som en användare utan inlogg.
\item[Galler och extrainstrument] De artiklar som benämns som galler och instrument bör också behandlas med en plocklista. Detta kan göras genom att lägga in dem i Kartoteket eller genom att skapa en separat databas för dessa med egen plocklista.
\item[Versionshantering av handböcker] Det finns ingen versionshantering av handböckerna. %Skriv mer utförligt
\item[Tillstånd av handböcker] Denna punkt hänger ihop med versionshanteringen. Vid användartester upplevdes tillstånden som något förrvirrande.
\item[Redigering av handböcker] I nuvarande system kan flera personer redigera en handbok samtidigt vilket kommer skapa problem med att personerna sparar över varandras ändringar. Ett lås skulle kunna implementeras för att lösa detta.
\item[Förhandstitt] I nuvarande system öppnas en ny flik vid förhandstitt. Från denna förhandstitt kan man gå vidare till redigering. Vid användartester upplevdes detta som något förvirrande. Redigeraknappen på förhandstitten bör alltså tas bort.
\item[Borttagning av handbok] I dagsläget går det inte ta bort en handbok. Detta bör vara enkelt att implementera.
\item[Handbok till pdf] Implementera så att pdf-filerna skapas direkt då en operation sparas. Detta kan göras med hjälp av ett api som redan är utvecklat. %skriv vad som kan ändras
\item[Redigering av Kartotekartikel] Det är svårt att redigera ett fält i Kartoteket som är tomt. Själva rutan är för liten och det är svårt att pricka rätt.
\item[Varning vid formulärifyllning] Det går inte att spara en artikel i Kartoteket då pris inte angivits. Användaren får dock ingen varning eller information om detta vilket leder till frustration.

\end{description}

\newpage
\part{Enskilda utredningar}
\renewcommand{\thesection}{\Alph{section}}	
\section{Daniel Rapp}
\subsection{Inledning}
Idag, när Region Östergötland tar hand om en patient i behov av en operation, så använder de sig av pappersbaserade handböcker och checklistor för att
förbereda operationen och ta fram rätt operationsutrustning.

\subsubsection{Syfte}
Syftet med projektet är att bygga en web-baserad, mobilanpassad, prototyp som ersätter, och förbättrar, det pappersbaserade systemet.


\subsubsection{Frågeställning}
Frågeställningar:
\begin{itemize}
  \item Kan man implementera ett kartotekssystem som uppfyller kundens önskemål?
\end{itemize}


\subsubsection{Avgränsningar}
Förutom ett förbättrat avcheckningsssytem så är Region Östergötland också i behov av
ett bättre system för att hantera deras lager på ett mer automatiserat sätt%, med t.ex. en
%scanner för att scanna av artiklar istället för manuellt checka av dem.
Bland annat
så skulle de behöva ett system som låter dem checka in vilka varor från lagret de hämtat
ut, istället för att checka av manuellt, vilket kan vara felbenäget.

Vi har dock valt att avgränsa oss från att bygga denna lösning, på grund av tidsbrister.


\subsection{Bakgrund}


\subsection{Teori}
%Agilt

\subsection{Metod}
\subsection{Resultat}
\subsection{Diskussion}
\subsubsection{Resultat}
\subsubsection{Metod}
\subsection{Slutsatser}

\newpage
\section{Node.js - Jonas Andersson}
\subsection{Inledning} 

Tidigare när jag har utvecklat webbprogram så har jag använt mig av PHP tillsammans med HTML, CSS och Javascript. Jag har även valt att skriva det mesta själv och inte använt mig av så mycket ramverk och bibliotek. I detta projekt valde jag, som arkitekt, att byta ut PHP mot Node.js och att istället för ren HTML och CSS använda oss av Handlebars och Less. Dessutom har jag lagt mycket vikt på att använda många andra bibliotek och ramverk för att slippa uppfinna hjulet på nytt.

\subsubsection{Syfte}

Syftet med denna del av rapporten är att analysera om vad det finns för fördelar och nackdelar med att använda Node.js till ett sådant här projekt. Det ska även undersökas vad det finns för fördelar och nackdelar med att använda många externa ramverk och bibliotek.

\subsubsection{Frågeställning}
Frågeställningar
\begin{itemize}
	\item Vad finns det för fördelar/nackdelar med att använda Node.js till detta projekt?
	\item Vad finns det för fördelar/nackdelar med att använda så många externa bibliotek som vi gjorde i detta projekt?
\end{itemize}
\subsubsection{Avgränsningar}
Det finns många programmeringsspråk som man kan använda i samma syfte som Node.js. Eftersom jag enbart har tidigare erfarenheter av PHP och Python så kommer alla jämförelser av Node.js vara mot dessa och inga andra. Exempelvis vägs fördelar och nackdelar mot PHP och Python.
\subsection{Bakgrund}

I detta projekt har jag haft rollen som arkitekt och alltså tagit fram grundstenarna till programmet. Jag valde en arkitektur byggd på Node.js med många bibliotek och ramverk. Node.js valdes framförallt av två anledningar. 

Först och främst så kom alla in i projektet med olika erfarenheter och kunskaper inom webbprogrammering och vissa hade aldrig tidigare skapat en hemsida. Trots detta ville vi snabbt komma igång och jag ville därför ta fram en arkitektur som var enkel att lära sig. Eftersom man på ett eller annat sätt är tvungen att skriva Javascript när man programmerar till webben så tyckte jag det var bra att hålla sig till så få nya språk som möjligt.

Den andra anledning till att jag valde Node.js beror på uppgiften. Det fanns ett krav som gick ut på att flera personer skulle kunna kryssa av en lista samtidigt och alla som var inne på sidan skulle se uppdateringen i realtid. Detta kan vara svårt att lösa på många plattformar men blir väldigt enkelt med websockets som Node.js har bra stöd för. Detta till skillnad mot exempelvis PHP där websocket är krångligare att använda.

I och med att jag inte hade så mycket erfarenhet av Node.js och att använda många bibliotek sedan tidigare så var det svårt att veta om det skulle passa bra till detta projekt. Dessutom hade jag ingen tidigare erfarenhet av webbprogrammering i grupp. Under projektets gång har jag samlat på mig mer erfarenheter som underlättar för att göra en bedömning av detta arkitekturval.

\subsection{Teori}

Nedan följer lite information om PHP, Python och Node.js samt vad som skiljer dom åt i stora drag.

\subsubsection{PHP}

PHP är ett programmeringsspråk som man kan skriva direkt i sin HTML-kod. Det skapades från början för att göra det enklare att lägga in funktionalitet på en hemsida. Du behövde inte längre skapa ett helt program i C, eller liknande, som i sin tur genererade HTML-kod för att exempelvis skapa en enkel gästbok\cite{php_history}. Nu kunde du enkelt lägga in PHP-kod direkt i HTML-koden som kördes i samband med att klienten frågar efter sidan. Det saknar även en standardiserad pakethanterare vilket gör att det är svårare att använda sig av många bibliotek. 

\subsubsection{Python}

På senare tid har flera språk som inte är skapade för webben från början blivit populära för att göra hemsidor. Ett exempel är Python vilket alltså inte är byggt för webben till en början men ramverk som exempelvis Flask\cite{python_flask} gör det möjligt att starta en enkel webbserver som genererar HTML-kod. Till skillnad från PHP så separerar alltså Python logiken från själva HTML-koden och man skriver inte Python direkt i HTML-filerna. Programmeringsstilen är ganska lik Node.js och man använder sig ofta av liknande designmönster, men eftersom språket inte är skapat för webben från början kan det få problem med exempelvis prestanda när antalet användare ökar.

\subsubsection{Node.js}
Internet ändras hela tiden. Om man går tillbaka några år så bestod mest internet av statiska hemsidor med enkla gästböcker eller liknande. Idag består internet av sociala nätverk och sidor med dynamiskt innehåll som kan ha tusentals användare varje dag. Node.js är skapat med detta i åtanke. Det är inte skapat för att göra tunga beräkningar men däremot för att snabbt kunna distribuera information till flera användare samtidigt. En stor skillnad på internet idag gentemot när PHP skapades är att när man har laddat klart en hemsida så kan fortfarande delar av hemsidan uppdateras. Exempelvis kan man på många sidor direkt se nya kommentar utan att uppdatera sidan eller som i detta projekt så ska flera personer kunna se när man kryssar i en checkruta i realtid. Detta är fullt möjligt i PHP och Python också men skillnaden är att Node.js är skapat för att lösa just sådana problem med bra prestanda. Detta har lett till att Node.js snabbt har fått stöd från många programmerare\cite{node_performance}.

\subsection{Metod}

I takt med att vi kom igång med projektet så fick vi i gruppen hela tiden nya erfarenheter med Node.js. Vi har upptäckt både fördelar och nackdelar med arkitekturen som jag inte hade tänkt på tidigare. 

Utifrån mina iakttagelser av gruppen tillsammans med mina egna åsikter har jag tagit fram vad jag ser för fördelar och nackdelar med Node.js och många bibliotek i detta projekt. Jag har även utifrån tidigare erfarenheter tagit med några potentiella fördelar och nackdelar som kan uppstå med denna arkitektur i längden.

Till sist har jag utifrån mina tidigare erfarenheter vägt fördelarna och nackdelarna gentemot hur det kan fungera i PHP och Python. 

\subsection{Resultat}

Under denna rubrik presenteras resultatet av undersökningen. Resultatet bygger på mina erfarenheter under projektets gång samt mina iakttagelser när jag läst andras kod i projektet.

\subsubsection{Fördelar med Node.js}

Först och främst var det en stor fördel med en plattform som är byggd runt sin pakethanterare, i detta fall npm. Det var väldigt enkelt att lägga till, uppdatera och ta bort bibliotek. När man behövde programmera under en ny miljö, exempelvis en annan dator så var det enkelt att bara behöva skriva en rad i kommandotolken eller liknande för att installera allt man behövde för att köra programmet. Det var även enkelt för alla att installera de moduler som behövs när någon valde att använda ett bibliotek.

Det var även lätt att hitta en bra grund att bygga på. Då vi det finns så många bibliotek var det enkelt att hitta ett kraftfull men utbyggbart system att börja med. I detta fall föll valet på KeystoneJS. Detta gjort att vi snabbt kom igång och inte behövde lösa allt med kodstruktur och liknande innan vi såg resultat.

En annan fördel som också hör till att det finns så mycket bibliotek att välja mellan är att det var väldigt enkelt att jobba realtidsuppdatering. Detta beror inte enbart på att vi hittade ett bra bibliotek, Socket.IO, utan även på hur språket är uppbyggt. Node.js är skapat för att klara av många småberäkningar samtidigt och det är precis vad det handlade om i detta projekt. Flera personer ska kunna kryssa i rutor och alla ska se vad som händer direkt. Det handlar inte om speciellt mycket datorkraft utan mer om många små meddelanden som ska skickas mellan klienterna.

Den sista fördelen med Node.js i just detta projekt var att man skriver i språket Javascript. Hade jag valt en annan plattform hade de som inte hållit på med webbprogrammering tidigare både behövt lärt sig Javascript för webbläsaren och ett nytt språk till servern. Nu räckte det med att lära sig Javascript vilket gick snabbt för alla att lära sig. Dessutom underlättade detta att jobba med kommunikation mellan klienterna då man enkelt skrev samma, eller liknande kod, på både klient och server.

\subsubsection{Nackdelar med Node.js}

Det finns tyvärr även nackdelar med Node.js. Den första är att plattformen skapades för att användas tillsammans med dokumentdatabaser. Vissa av delarna i detta projekt skulle passa bättre att ha i relationsdatabaser. Det är visserligen fullt möjligt att med hjälp av bibliotek till Node.js använda sig av relationsdatabaser men det finns inte lika bra och utarbetat stöd som det gör för dokumentdatabaser.

En annan nackdel var att Node.js programmeringstänk var svårt att komma in i till en början. Tanken är att man all kod ska köras asynkront. I början var det lite ovant för många att skriva kod på det sättet. 

Eftersom både server och klient använder sig av Javascript så kunde det ibland vara svårt att hålla isär vad som händer på servern respektive klienten. Exempelvis läsning och skrivning till databasen kan enbart ske från servern och man måste därför använda sig av något gränssnitt som i sin tur körs på servern för att göra detta genom klienten. Detta var lite krångligt till en början och det var vanligt att man blandade ihop hur man pratar med databasen beroende på om man är på servern eller klienten.

Den sista nackdelen handlar om prestanda. I fall där tyngre och längre beräkningar behövdes göras så märks snabbt en fördröjning på klienten. Jag tänker specifikt på en redigeringssida i projektet. Sidan har relativt komplexa databaskopplingar och ifall man ville skicka all data och sedan tolka den på servern tog det mycket kraft och det var även svårt att skriva kod som kan köras asynkront. Därför var vi tvungna att flytta mycket av logiken till klienten istället.

\subsubsection{Fördelar med många externa bibliotek} 

Den kanske största fördelen med externa bibliotek är att man snabbt når resultat. Det går fort att lägga in ett paket som löser stora delar av ditt problem och eftersom det finns så många bibliotek till Node.js så går det ofta lätt att hitta ett som passar för just ditt problem.

Det blir också ofta i många fall snygga och mer utarbetade lösningar än om man skrivit det själv. Eftersom många bibliotek används av tusentals programmerare så utvecklas de även hela tiden. På det sättet har många bibliotek växt fram till att bli väldigt kraftfulla och effektiva. 

\subsubsection{Nackdelar med många externa bibliotek}

En stor nackdel som jag framförallt har upptäckt i tidigare projekt men även ett par gånger i detta projekt är att man lätt kan få konstiga buggar som är svåra att felsöka. I vissa fall är inte bibliotek kompatibla med varandra och om man då inte har lite tur med en Google-sökning och hittar någon med exakt samma problem så kan de vara väldigt svårt att hitta ursprunget till buggarna som uppstår. Detta kan även hända när man uppdatera ett befintligt bibliotek som i och med uppdateringen blir inkompatibelt med något annat bibliotek. På så sätt kan det enkelt eskalera ifall man använder sig av allt för mycket stora bibliotek.

Dessutom finns det även en säkerhetsaspekt som inte spelar så stor roll i detta projekt då programmet enbart körs på ett intranät men kan spela roll i framtiden. Vissa bibliotek lider av säkerhetsproblem som upptäcks och täpps till under utvecklingen. Med många bibliotek i ett projekt kan det vara svårt att hålla alla uppdaterade och man riskerar då att vara utsatt för en säkerhetsrisker.

En annan nackdel är att när man är ny inom webbprogrammering och inte har full förståelse för hur saker och ting hör ihop så kan det ibland kännas som saker händer av ren magi. Man ser att det fungerar men man vet inte hur. Om då till exempel ett testfall inte fungerar kan det vara svårt att lokalisera och lösa. Detta hände framförallt med det stora biblioteket KeystoneJS som gör mycket i bakgrunden och lyfter fram ett enkelt gränssnitt till programmeraren.

Den sista nackdelen är att hålla reda på alla licenser. Det finns en uppsjö av licenser för öppen källkod och det gäller att ha koll på vad man får, och inte får, göra med koden. När dessutom alla kan lägga till bibliotek hur de vill så kan det lätt slinka igenom ett bibliotek som har en otillåten licens.


\subsection{Diskussion}

\subsubsection{Resultat}

Många av fördelarna för tyvärr även med sig en nackdel. Så väger fördelarna verkligen över nackdelarna? 

Om vi börjar med att Node.js använder sig av Javascript så tycker jag det är en väldigt stor fördel när man börjar med en grupp som är ny inom webbprogrammering. Det är inte så stort att lära sig ett nytt språk men kan bli krångligt att lära sig två språk samtidigt. Man blir lätt att blanda ihop små finesser i språken och dessutom hinner man inte lära sig två språk lika djupt. Samtidigt skapade Javascript förvirring mellan vad som kördes på servern respektive klienten. Det hade varit tydligare ifall man använt sig av ett helt annat språk på servern, som exempelvis PHP eller Python.

Men med både PHP och Python blir det istället krångligare med kommunikationen mellan server och klient. Det som nu kunde skriva med samma, eller liknande, kod på både server och klient skulle kunna behöva varit helt annorlunda. Säkert är i alla fall att det inte skulle gått att kopiera, eller flytta kod mellan server och klient.

Resterande nackdelar med Node.js tycker jag inte har någon större betydelse i detta projekt då det handlar om ett relativt litet projekt. Saker som prestanda kommer inte bli lidande i nuläget. Däremot ifall programmet ska skalas upp och köras på flera sjukhus ifrån samma instans skulle man behöva undersöka detta närmare. Med det inte sagt att en sådan undersökning skulle visa att PHP, Python eller någon annan plattform hade varit ett bättre alternativ.

När det gäller att använda många externa bibliotek så är det enkelt att välja en arkitektur som ger snabba resultat framför en som ger mer långsiktiga lösningar i ett sådant här projekt där man har begränsat med tid. I ett projekt som pågår under längre tid kan det vara bättre med en lite mer konservativ inställning kring att ta in nya bibliotek för att lösa ett problem. I många fall är det bra och man får enkelt effektiva och kraftfulla lösningar men ibland gör det också projektet mer svårhanterligt. Som jag skrev i resultatet så finns det en risk att buggar uppkommer på grund av inkompatibilitet mellan bibliotek. 

Med det inte sagt att man inte bör använda externa bibliotek överhuvudtaget men det bör kanske ligga lite mer forskning bakom innan man väljer att ta inte ett bibliotek i ett projekt. I detta projekt har vi tagit in ett projekt, kollat så det löser vårat problem och att licensen är tillåten och ifall detta är uppfyllt så har vi behållit biblioteket. Här finns det utrymme för förbättring och mer forskning. Exempelvis kan det vara bra att undersöka ifall biblioteket är stabilt, effektivt och vad som finns för support om något inte fungerar. Det kan även vara bra att sätta sig in lite i koden för att se så den håller måttet.

\subsubsection{Metod}

Det finns mycket jag skulle kunnat gjort för att få bättre resultat i denna rapport. Exempelvis skulle en undersökning kunnat gjorts i gruppen för att se vad de tyckt varit bra respektive dåligt med egna ord och inte bara gått på mina iakttagelser. Jag hade även kunnat hitta mer källor på nätet om vad som generellt är bra och dåligt med Node.js och utvärderat hur det har påverkat vårat projekt. 

Det går även att tänka ur ett mer långsiktigt perspektiv där kunden tagit över utvecklingen. Helt enkelt hur det skulle fungera att ta över detta projekt och koppla detta till min frågeställning. Jag tänker då på kunden som väljer mellan att modifiera detta projekt eller låta ett företag skriva om från början. En undersökning ur ett långsiktigt perspektiv skulle kunna hjälpa kunden i detta beslut.

\subsection{Slutsatser}

Just för detta projekt tycker jag valet av Node.js och att använda många bibliotek var bra. Vi hade press på oss att snabbt få en färdig produkt och att använda mycket färdig kod har hjälpt oss dit. 

De flesta delarna skulle nog kunna göras lika bra i PHP eller Python men det hade förmodligen varit en något längre inlärningskurva för gruppens medlemmar. Jag tror även att just delen med att flera personer samtidigt ska kryssa av en lista skulle varit svårare att lösa med PHP eller Python och är ett problem som egentligen är perfekt för Node.js.

Jag tycker att fördelarna väger över nackdelarna för både Node.js och för att använda många bibliotek i detta projekt.

\subsection{Referenser}
\vspace{-9mm}
\begin{thebibliography}{9}

	\bibitem{php_history}
	\url{http://php.net/manual/en/history.php.php}\\
	Hämtad 2015-05-11.
	
	\bibitem{python_flask}
	\url{http://flask.pocoo.org/}\\
	Hämtad 2015-05-11.
	
	\bibitem{node_performance}
	\url{http://arxiv.org/abs/1503.01398?}\\
	Hämtad 2015-05-12.
	
\end{thebibliography}
\newpage
\section{Pål Kastman}
\subsection{Inledning}
I detta projekt så har vi valt att inledningsvis skriva dokumentationen i Googles ordbehandlingsprogram Google Docs för att 
i ett senare skede då det var dags att påbörja denna rapport gå över till att använda typsättningssystemet Latex. 

Vi valde att göra på detta sätt för att så snabbt som möjligt komma igång, då man bland annat i Google Docs kan se live 
vad andra skriver.

\subsubsection{Syfte}
Syftet med denna individuella del är att undersöka funktionaliteten i Latex och väga fördelar mot nackdelar.

\subsubsection{Frågeställning}
\begin{itemize}
\item Vad finns det för begränsningar i Latex
\item Kommer det att vara en fördel eller en nackdel för flera gruppmedlemmar att jobba samtidigt i samma delar av rapporten.
\end{itemize}

\subsubsection{Avgränsningar}
Denna rapport kommer att avgränsas för att endast jämföra Latex, Microsoft Office och Google Docs, detta för att 
inte behöva jämföra alla olika ordbenhandlingsprogram.

\subsection{Bakgrund}
Dokumentering är någonting som är viktigt att göra när man arbetar i projektform, dels för egen del utifall man behöver 
gå tillbaka och se vad som gjorts, men även för andras skull ifall man kanske får en ny medarbetare som ska integreras i 
projektet.

En fråga som man alltid behöver besvara är i vilket ordbehandlingsprogram dokumentationen skall skrivas. Det populäraste 
alternativet kan tänkas vara Microsoft Word vilket har funnits sedan 1983.\footnote{http://ia801406.us.archive.org/21/items/A\_History\_of\_the\_Personal\_Computer/eBook12.pdf Hämtad 2015-04-20.}.

Ett annat alternativ som blir allt vanligare är Google Docs.

Ett tredje och inte lika vanligt alternativ är Latex. Detta är egentligen bara en vidareutveckling av typsättningssystemet 
Tex vilket skapades av den amerikanske matimatikern Donald Knuth 1978\footnote{https://gcc.gnu.org/ml/java/1999-q2/msg00419.html Hämtad 2015-04-20.}. Leslie Lamport var mannen som vidareutvecklade detta 
språk i början av 80-talet, till det vi idag kallar Latex\footnote{http://research.microsoft.com/en-us/um/people/lamport/pubs/pubs.html\#latex Hämtad 2015-04-20.}.




\subsection{Teori}
När Microsoft utvecklade Word på 80-talet så ville de naturligtvis att dåtidens datorer skulle klara av att ladda 
dokument utan att det skulle vara alltför krävande, därför konstruerade man dess filformat (.doc) binärt och gjorde konstruktionen 
väldigt komplex. Detta medförde att bara personer som var riktigt insatta i detta filformat klarade att ändra direkt i dessa filer.

Tex däremot är i grunden ett märkspråk liksom t.ex. HTML, där man istället ändrar direkt i koden genom att tilldela text olika taggar

\subsection{Metod}
Vi har till en början valt att under projektets gång arbeta i endast en fil för den gemensamma delen av kandidatprojektet. Vad det gäller 
de individuella delarna så har vi valt att lägga dessa i separata filer och sedan importera dessa till den gemensamma. Vi 
sparar alla dokumentfiler i samma repository på github som vi har källkoden.

Genom att göra på detta sätt så kan vi garantera att det inte kommer att uppstå några konflikter i de individuella delarna.
Det som vi inte vet, är huruvuda det kommer att uppstå problem då alla gruppmedlemmar skall skriva på den gemensamma delen.

\subsection{Resultat}
\subsection{Diskussion}
\subsubsection{Resultat}
\subsubsection{Metod}
\subsection{Slutsatser}
\subsection{Referenser}
\begin{enumerate}
\item http://ia801406.us.archive.org/21/items/A\_History\_of\_the\_Personal\_Computer/eBook12.pdf Hämtad 2015-04-20.
\item https://gcc.gnu.org/ml/java/1999-q2/msg00419.html Hämtad 2015-04-20.
\item http://research.microsoft.com/en-us/um/people/lamport/pubs/pubs.html\#latex Hämtad 2015-04-20.
\end{enumerate}
\newpage
\section{Kravinsamlingsmetoder - Daniel Falk}
\subsection{Inledning}
Jag har i detta projekt haft rollen som analysansvarig vilket innebär en analys av kundens behov och framställning av krav utifrån dessa. Denna enskilda del beskriver vilka kravinsamlingsmetoder och verktyg som använts för att ta fram krav. Erfarenheterna används för att undersöka om metoderna och verktygen kan användas för att tillfredsställa kundens verkliga behov.
\subsubsection{Syfte}
Syftet med denna del av rapporten är att undersöka olika metoder och verktyg för kravframställning. Fokus ligger på intervjuer, observationer och prototyper. Rapporten går speciellt in på hur prototyper använts för att ta fram, testa och validera krav i detta projekt.
\subsubsection{Frågeställning}
%Hur kan man samla in krav som tillfredställer kundens verkliga behov?
\begin{itemize}
\item Är intervjuer och observationer bra metoder för att analysera kundens behov?
%\item Hur kan intervjuer och observationer användas för att analysera kundens behov?
\item Hur kan man arbeta med prototyper för att framställa krav?
\item Hur kan man använda prototyper för att testa och validera krav?
\end{itemize}
\subsubsection{Avgränsningar}
Rapporten har sin utgångspunkt i hur kravframställningen gått till i detta projekt och gör inga jämförelser med andra kravinsamlingsmetoder. Metoden bör inte ses som ett allmänt tillvägagångssätt.
\subsection{Bakgrund}
Att förstå kundens verkliga behov utgör grunden för ett lyckat projekt. Ett projekt faller ofta på grund av ofullständiga krav \cite{Hull}.
För att samla in krav kan flera olika metoder och tekniker användas. Metoderna kan variera och vara olika bra på att fånga upp olika typer av krav.
\subsection{Teori}
Teorin beskriver kravinsamlingsmetoder och  verktyg som använts i detta projekt. %och vilka problem som kan uppstå vid kravframställning. 
%\subsubsection{Problem}
%Skriv om och flytta till bakgrund
%Det finns ett antal barriärer som kan uppstå vid kravframställning. Ett problem är ofta att kunden inte kan formulera vad de vill ha. De kan se problemet men inte vad som behöver göras. De kan överdriva vissa problem medan andra förbises. 
%Ett annat problem är att kunden kan ha fastnat i ett invant mönster. De kan ha svårt att föreställa sig nya sätt att utföra en uppgift på.
%\cite{Lauesen} %Kolla upp om denna parafras är ok.

\subsubsection{Prototyper}
Prototyp är ett ord som har sina rötter i grekiskan och betyder \textit{första form} \cite{Arvola}. Deras syfte är att i ett tidigt skede beskriva hur det färdiga systemet ska fungera. Prototyper kan användas för att testa olika ideér och designer och kan variera i detaljrikedom. 

En tydlig skillnad är den mellan enkla LoFi-prototyper skissade på papper och datorbaserade HiFi-prototyper som mer liknar det riktiga systemet. LoFi-prototyper är ett bra verktyg för att snabbt kunna diskutera ett designval då det kräver väldigt lite arbete. En styrka hos LoFi-prototyper är att användare har lätt att komma med kritiska kommentarer utan att känna att de förolämpar designern \cite{Arvola}.

Prototyper kan vara temporära eller evolutionära \cite{Arvola}. En temporär prototyp är en prototyp som slängs efter att man har använt den och utvärderat den. En evolutionär prototyp slängs inte utan byggs vidare på. Man kan se det som en tidig version av det slutgiltiga systemet.

\subsubsection{Intervjuer}
%För att samla in krav kan flera olika metoder och tekniker användas. Metoderna kan variera och vara olika bra på att fånga upp olika typer av krav.
%Intervjuer
Intervjuer är en bra teknik för att få information om det nuvarande arbetet inom området och problem relaterade till det. Det är också bra för att få fram de stora målen med ett projekt. Många ser det som den huvudsakliga insamlingstekniken. De ger mycket information men för att lösa kritiska problem behövs ofta andra tekniker för att komplettera intervjuer \cite{Lauesen}. 
\subsubsection{Observationer}
Observationer är ett bra verktyg för att få information om nuvarande system eller arbetssätt. Det är ett bra sätt att komplettera intervjuer då användare ofta har svårt att förklara vad de verkligen gör \cite{Lauesen}. 
%(task demonstration)
%(scenarios)
 
\subsection{Metod}
Denna del beskriver hur intervjuer, observationer och prototyper har använts som metoder för kravframställning i detta projekt. Metoden beskriver också hur systemet testats för att validera och testa kraven. Frågeställningarna besvaras utifrån denna erfarenhet.
%Denna del beskriver hur kravframställningen gått till i detta projekt. Först beskrivs arbetet med att analysera kundens behov under förstudien. Vidare beskrivs mer ingående vilka metoder som använts och hur kraven valdes att representeras. Avslutningsvis beskrivs vilka användartester som genomfördes och hur dessa bedrog till utvecklingen.
%\subsubsection{Förstudie}
%Den största delen utav analysarbetet skedde under förstudien. Här identifierades de olika intressenterna och en kravspecifikation utarbetades. Våran första kontakt med projektet var en projektbeskrivning där kunden formulerade sina mål och visioner av projektet. Några vikta krav gavs också såsom att prototypdesign skulle genomföras i samarbete med kunden. Ett första möte utav fyra under förstudien gav sedan mer information och vi började våran kravinsamling. Vi kunde konstatera att vi hade två olika intressenter att arbeta med. Dels sjuksköterskorna som är användare av systemet och dels CMIT, Centrum för medicinsk teknik och IT, som ansvarar för sjukhusets IT-miljöer. För att förstå användarnas behov hölls ett studiebesök där vi fick en visning av nuvarande system och hur det används. Detta var nödvändigt för att verkligen förstå vad det var som behövde göras och vad som kunde förbättras. Från CMIT:s sida hölls mer tekniska möten där teknikval diskuterades. Här var det viktigt att ta reda på vilka begränsningar som fanns och vilka val som passade våra och deras erfarenheter.
\subsubsection{Intervjuer}
Möten med kund har genomförts vilka kan ses som intervjuer. Dessa möten har skett dels med IT-ansvariga, verksamhetsutvecklare och sjuksköterskor. För att samla in krav under dessa möten har vi dels fört anteckningar på papper eller dator och dels spelat in på mobil. När vi varit flera personer på möten har vi gått igenom och diskuterat våra anteckningar i efterhand. Vid oklarheter har vi antecknat dessa för att förtydliga med kund. Inspelning användes främst vid första mötet och var användbart då vi fick mycket information att sätta oss in i. 
%Kravspecifikationen skrev i Google docs. Detta valdes eftersom kraven utarbetades tillsammans med kund. Ett gemensamt redigerbart dokument gav en möjlighet för oss att arbeta på olika platser under kravframställningen vilket var effektivt. En kommentarsfunktion gav oss möjligheten att kommentera krav och föreslå förbättringar. Under interna möten och kundmöten var den gemensamma redigeringen också till nytta då kravformuleringar snabbt kunde genomföras.

%\subsubsection{Kravrepresentation}
%Kraven gavs en prioritetsordning för att kunna urskilja de mest väsentliga kraven. Detta kändes nödvändigt då vi var begränsade av en tidsbudget. En prioritering av kraven gav utrymme för vidareutveckling i mån av tid. Krav med prioritet 1 var att betrakta som grundkrav som skulle genomföras för att projektet skulle ses som godkänt. Krav med prioritet 2 var att betrakta som önskvärda och som skulle genomföras om då grundkraven var genomförda. Krav med prioritet 3 var krav som fångats upp med som skulle ses som framtida utbyggnad. 
%Koppla till nån litteratur/standard.

%Strukturering av krav
%Kraven numrerades för att lätt kunna refereras till under projektets gång. De delades också in i olika sektioner efter deras del i systemet. Sektionerna var plocklistor, handböcker, kartotek, lagersystem. En extra sektion för generella krav användes. Denna uppdelning kändes naturlig för detta projekt. 
%Man skulle också kunna dela in dem efter blablabla...

%Kravspecifikationen skrevs med stöd från standarden IEEE 830. Enligt standarden ska ett krav vara korrekt, otvetydigt, färdigt, konsekvent, prioriterat, verifierbart, modifierbart och spårbart. Detta eftersträvades men det kan diskuteras om alla krav passerar dessa filter. I slutändan var det ändå våran gemensamma förståelse för kravet tillsammans med kunden som accepterades. 

%Enligt standarden uttrycktes kraven på ska-form. Ett exempel på ett krav från projektet är följande: "Plocklistor ska innehålla information om artikelns namn, förråd, sektion, hylla och fack".

\subsubsection{Observationer}
För att få en inblick i hur operationsförberedelserna fungerar i dagsläget genomfördes ett studiebesök på sjukhusets operationsavdelning. Vi fick här se problemet ur sjuksköterskornas synvinkel. Vi fick se hur man arbetade i artikellagret i dagsläget vilket var nödvändigt för att kunna genomföra en förbättring. De svårigheter som beskrivits blev tydligare och det blev lättare att föreställa sig en lösning på problemet.

\subsubsection{LoFi-prototyper}
Under förstudien valde vi att vid två tillfällen göra LoFi-prototyper som vi tog med till kund. Vid dessa tester kunde vi se om vi var på rätt bana när det gällde design och struktur. LoFi-protptyperna utjordes av enkla pappersskisser. Ett exempel visas i figur~\ref{fig:lofiprototyper}.
\begin{figure}[htbp]
\begin{center}
\includegraphics[scale=0.2]{lofiprototyper.png}
\caption{LoFi-prototyper}
\label{fig:lofiprototyper}
\end{center}
\end{figure}

Det första tillfället fokuserade på gränssnittsdesign och navigation i systemet. Prototyperna visades för en sjuksköterska som fick föreställa sig systemet. De olika korten gicks igenom för att se om vi hade hittat en bra design och om navigationen var logisk.

Vid det andra tillfället hade vi en intern brainstorming för att ta fram en design för redigeringsvyn av en handbok. Två olika förslag togs fram och presenterades för kunden. Vi observerade och antecknade försökspersonernas olika tankar samt spelade in en ljudupptagning på mobiltelefon för att kunna gå tillbaka och analysera.
%Note to self: Lyssna på det ljudklippet.
\subsubsection{HiFi-prototyp}
Själva systemet som vi utvecklat kan ses som en HiFi-prototyp. Den är evolutionär på så sätt att den byggts vidare på under varje iteration. Huruvida kunden kommer fortsätta utvecklingen av denna prototyp på egen hand är inte bestämt. Ett alternativ är att se vårt system som en prototyp till ett nytt system. 
Prototypen presenterades och utvärderades i samband med kundmöten vid varje iterationsslut. Ett mer omfattande användartest utfördes i iteration 3.


%\subsubsection{Kravvalidering}
%Vid varje iterationsslut hölls ett möte med kund där vi demonstrerade nya features och lät kunden testa systemet. Iterationerna gjorde att vi snabbt kunde rätta till eventuella missförstånd. Vid dessa möten diskuterade vi också vad som skulle genomföras nästa iteration. Vi gick igenom vilka krav som var genomförda och vilka som skulle prioriteras till nästa iteration.

\subsubsection{Användartest}
I iteration 3 av projektet genomfördes användartester av systemet vid två tillfällen. Vid dessa tester kördes det nya systemet parallellt med det gamla. Testen fokuserade på skapande av handböcker och genomförande av operationsförberedelser. Testen genomfördes dels på egen hand av verksamheten och dels med delar av projektgruppen på plats. Vid användartesterna fick vi in feedback från användarna om vad som var bra och vad som kunde göras bättre. Ett exempel på designval som kunde utredas vid testningen var om en operationsförberedelse skulle kunna sättas som klar även om inte alla artiklar var plockade. Olika alternativ hade diskuterats innan men vid testning blev det tydligt att det var en önskvärd funktionalitet. Andra exempel på saker som upptäcktes vid användartesten var hur engångsmaterialet skulle sorteras på bästa sett,hur navigationen mellan olika sidor kunde fungera effektivare och att vissa saker inte användes.
Systembuggar kunde också rapporteras och åtgärdas under testningen.
Kunden höll också en intern utvärdering om hur användarna upplevde systemet (se Appendix A).

%skriv eventuellt om de övriga artiklarna som inte behandlas under engångsmaterial

\subsection{Resultat}
Att använda intervjuer och observationer som metoder för kravinsamling kändes naturligt för detta projekt. Det var ett bra tillvägagångssätt för att få en helhetsbild av vad kunden ville ha. Intervjuerna gav oss kundens mål och visioner av projektet. Observationerna gav oss en inblick i hur användarna använde systemet och saker som var otydliga i intervjuerna kunde fångas in i observationen. På så sett kompletterade dessa metoder varandra och det kändes nödvändigt för en lyckad kravinsamling.

LoFi-Prototyper kunde användas i förstudien för att bekräfta våran bild av vad som skulle byggas. De var ett bra verktyg för att reda ut otydligheter. De var effektiva på så sätt att de gick snabbt att tillverka och gav mycket feedback tillbaks. Det kändes lättare att diskutera systemet när vi hade något visuellt framför oss. De gjorde det lättare att formulera krav och bidrog på så sätt till kravinsamlingen.

HiFi-prototypen kunde användas för att testa systemet utifrån kraven. Vid varje iterationsmöte kunde systemet utvärderas och kraven kunde både formuleras om och prioriteras om. Vi kunde också använda prototypen för att validera vilka krav som var genomförda. 
 
Vid användartesterna i iteration 3 kunde systemet användas för att testa om kraven uppfyllts och formulerats efter kundens behov. Här framkom vissa förändring på krav såsom att mer information skulle visas för en artikel. Prototypen kunde på så sätt användas för att testa och validera krav. För en eventuell vidareutveckling kan observationerna användas som underlag för att skriva nya krav.
%TODO: Fyll på med resultat från användarutvärdering

\subsection{Diskussion}
Här diskuteras resultatet av den enskilda utredningen och hur metoden fungerat för att undersöka frågeställningarna.
\subsubsection{Resultat}
Resultatet bekräftar synen på vad som sägs i teorin. Intervjuer och observationer kunde användas som komplement till varandra på ett bra sätt. Det kan diskuteras om dessa metoder är tillräckliga för en kravinsamling. För ett projekt i en storlek som vårt kan de vara tillräckliga. 

Prototyper användes i vårt projekt som ett verktyg för att samla in krav. Andra verktyg kunde dock ha använts för att ytterligare komplettera dessa metoder. 

Vi la inte så stort fokus på systemets olika roller i projektet utan nöjde oss med två stycken. För att analysera de olika rollerna kunde verktyg som user-stories ha använts.
Prototyper kan se olika ut och användas på många olika sätt. I vårt projekt visades att de var bra för att samla in krav och snabbt reda ut otydligheter. 
\subsubsection{Metod}
Metoden som användes för att svara på frågeställningarna var att utföra själva projektet och utgå ifrån dessa erfarenheter. Projektet gav således svar på att intervjuer, observationer och prototyper gick att använda och var bra för just detta projekt. Metoden har inte gjort några jämförelser till andra kravinsamlingsmetoder vilket skulle kunna vara intressant. 
%Huruvida metoden passar andra projekt kan diskuteras. 

\subsection{Slutsatser}
Intervjuer och observationer är bra grundmetoder för att analysera kundens behov. I detta projekt kändes de som naturliga och nödvändiga tillvägagångssätt. De kan med fördel kompletteras med olika verktyg och tekniker såsom prototyper. 

Prototyper är ett väldigt kraftfullt verktyg för kravinsamling och kan användas på flera olika sätt. LoFi-prototyper är bra i början av ett projekt för att snabbt kunna kommunicera ideér och hitta en bra design. De hjälper på så sätt till i kravinsamlingsprocessen. HiFi-prototyper är bra både för att testa om kraven formulerats på ett bra sätt och för att validera systemets funktionalitet.
\subsection{Referenser}
%Note to self:Dubbelkolla referenserna, tror det är fel standard

\vspace{-9mm}
\begin{thebibliography}{9}
\bibitem{Hull}
Hull. E, Ken. J och Jeremy. D, Requirements Engineering, Third edition. London: Springer, 2011.
\bibitem{Lauesen}
Lauesen, S. (2002) Software Requirements: Styles and Techniques. Harlow: AddisonWessly.
\bibitem{Arvola}
Arvola, M. (2014) Interaktionsdesign och UX: Om att skapa en god användarupplevelse. %Kolla upp förlag!!!!!!!!
\end{thebibliography}


\newpage
\section{Automatiserade tester av webbapplikationer. - Erik Malmberg}
\subsection{Inledning}
Den här enskilda utredningen är en del av kandidatrapporten i kursen TDDD77 vid Linköpings universitet.
Utredningen behandlar en del av utvecklingen av ett webbaserat system för att underlätta förberedelser
inför operationer på sjukhusen i Östergötland. Systemet utvecklades på uppdrag av Region Östergötland.

\subsubsection{Syfte}
Syftet med den här enskilda delen av kandidatarbetet är att ge insikt 
i hur kontinuerlig integration och 
automatiserade tester kan användas för att effektivisera testandet 
i ett projekt som använder en agil 
utvecklingsmetod. Speciellt ska det undersökas hur väl det går att använda 
webbaserade tjänster för
att utföra kontinuerliga automatiserade tester av webbapplikationer.

\subsubsection{Frågeställning}
De frågeställningar som ska besvaras i den här enskilda 
delen av rapporten är:

\begin{itemize}
\item Hur kan man använda webbaserade tjänster för
att utföra kontinuerliga automatiserade tester av webbapplikationer?
\item Hur effektivt är det att använda en webbaserad tjänst
för automatiserade tester? 
\item Vilka typer av tester är svåra att utföra med en sådan tjänst?
\end{itemize}

I den andra frågeställningen så definieras effektivitet som antalet test som kan
utföras per sekund. I svaret på frågeställningen ska även testfallen 
specificeras noggrant så att svaret inte blir tvetydigt. Det kan även vara
intressant att ta reda på hur lång tid det tar från det att koden läggs 
upp på GitHub
till det att testfallen börjar köras på servern.

\subsubsection{Avgränsningar}
Inga undersökningar kommer att utföras om hur andra lösningar än 
Travis CI kan användas för kontinuerlig 
integration och automatiserde tester. De testfall som kommer användas 
kommer uteslutande att vara skrivna med ramverket Jasmine. Den webbapplikation
som kommer att testas kommer att vara skriven med programmeringsspråket
Javascript och använda Javascriptbiblioteken Node.js och jQuery.

\subsection{Bakgrund}
Här beskrivs de tjänster, språk och bibliotek som använts under arbetet
med den här enskilda rapporten.

\subsubsection{Travis CI}
Travis CI är en webbaserad tjänst för att köra automatiserade enhetstester och integrationstester
på projekt som finns på GitHub. Travis CI är gratis att använda och byggt på öppen källkod
som är tillgänglig under en MIT-licens. 
Tjänsten har stöd för många olika programmeringsspråk, men det som är 
relevant för innehållet i den här rapporten
är Javascript med Node.js. För att konfigurera Travis CI används filen .travis.yml 
som placeras i det aktuella
projektets repository på GitHub.
Travis CI kör automatiskt de testfall som specificerats av användaren 
när kod läggs upp på GitHub.

\subsubsection{Javascript}
Javascript är ett programmeringsspråk som i första hand används på klientsidan på webbsidor.
Javascript exekveras av webbläsaren och arbetar mot ett gränssnitt som heter 
Document Object Model (DOM).

\subsubsection{JQuery}
JQuery är ett Javascriptbibliotek som kan användas för att förenkla programeringen
av Javascript på klientsidan av en webbsida. JQuery innehåller lättanvänd
funktionalitet för händelsehantering och modifiering av HTML-objekt.
JQuery är gratis och baserat på öppen källkod som är tillgänglig under en
MIT-licens.

\subsubsection{Node.js}
För information om Node.js se rubriken \emph{2.1.1 Node.js}
i det gemensamma avsnittet \emph{2.1 Programspråk och bibliotek}.

\subsubsection{Jasmine}
Jasmine är ett ramverk för testning av Javascript. 
Den node-modul som används är grunt-contrib-jasmine som använder task runnern Grunt 
för att köra testfall som skrivits med Jasmine.
Grunt kunfigureras med filen Gruntfile.js. Jasmine är baserat på öppen källkod
som är tillgänglig under en MIT-licens.

\subsection{Teori}
Här beskrivs den teori som ligger till grund för rapporten och som visar
varför frågeställningarna är relevanta.

\subsubsection{Vattenfallsmodellen}
I vattenfallsmodellen genomförs all integration och alla tester efter att implementeringen är slutförd. 
Om ett problem då identifieras under integrationen så är det krångligt att gå 
tillbaka och åtgärda problemet, vilket 
kan leda till förseningar av projektet.
Om felet som upptäcks är så allvarligt att en betydande omdesign måste ske så
kommer utvecklingen i stort sett att börja om från början och man kan räkna 
med en hundraprocentig ökning av budgeten, 
både vad gäller pengar och tid \cite{Royce}.

\subsubsection{Kontinuerlig integration och automatiserade tester}
Kontinuerlig integration kan leda till att problemen identifieras tidigare i 
utvecklingsprocessen. Problemen blir då lättare att åtgärda. Automatiserade tester kan effektivisera 
testprocessen och det finns många tillgängliga lösningar för att köra automatiserade
tester \cite{Karlsson}.
Några av de vanligaste är Travis CI, Codeship och Drone.io.

\subsection{Metod}
Här beskrivs den metod som användes för att 
besvara frågeställningarna.

Arbetet inleddes med att en teoriundersökning genomfördes
angående vilka metoder, ramverk och tjänster som skulle kunna
användas för att genomföra testerna i projektet.

De ramverk för
testning som undersöktes var Jasmine, Qunit och Mocha. Det
ramverk som valdes för arbetet med projektet var Jasmine
eftersom det är enkelt att konfigurera och har ett 
tydligt och intuitivt syntax. Att Jasmine var enkelt och tydligt
var viktigt eftersom det var många andra uppgifter förutom
testning som skulle utföras i projektet och tiden för
arbetet med projektet var begränsad till 300 timmar. Det var
alltså ingen i projektgruppen som hade tid att lära sig 
en komplicerad syntax i ett ramverk även om det varit
mer kraftfullt.

De tjänster för automatiserade tester som undersöktes var
Travis CI, Codeship och Drone.io. Även här 
så gjordes valet beroende på hur enkla tjänsterna var att
konfigurera och använda. Den tjänst som verkade enklast
och således valdes var Travis CI.

Arbetet med Travis CI inleddes med att tjänsten kopplades till 
projektets repository på GitHub. Kopplingen utfördes
genom att administratören för repositoryn loggade in på travis-ci.org med 
sitt GitHub-konto och aktiverade
en webhook för repositoryn.

Inställningarna för Travis CI konfigurerades med filen .travis.yml i projektets
repository. Språket valdes till
Javascript med Node.js med inställningen: \emph{language: node\textunderscore js}.
Versionen av Node.js valdes till version 0.10
med inställningen: \emph{node\textunderscore js: "0.10"}.

De nödvändiga node-modulerna installerades med hjälp av node package manager (npm).
Grunt installerades
med kommandot: \emph{npm install -g grunt-cli}. Grunt-contrib-jasmine installerades med kommandot: 
\emph{npm install grunt-contrib-jasmine}.

Task runnern Grunt konfigurerades med filen Gruntfile.js i projektets repository.
En task för Jasmine laddades in med
inställningen: \emph{grunt.loadNpmTasks('grunt-contrib-jasmine');}.
Tasken konfigurerades med följande kod i Gruntfile.js.

\begin{lstlisting}[
  basicstyle = \small
]
module.exports = function(grunt) {

  grunt.initConfig({
    pkg: grunt.file.readJSON('package.json'),
    jasmine: {
      test: {
        src: './public/js/*.js',
	options: {
	  vendor: [
	    'public/js/lib/jquery/jquery-2.1.1.js',
	    'node_modules/jasmine-jquery/lib/jasmine-jquery.js'
          ],
	  keepRunner: true,
	  specs: 'test/*-spec.js',
	  template: 'test/template/spec-template.tmpl'
        }
      }
    }
  });

  grunt.loadNpmTasks('grunt-contrib-jasmine');
}
\end{lstlisting}

Med \emph{src: './public/js/*.js'} valdes de filer som som skulle testas.
Med vendor valdes andra filer som var nödvändiga för att köra testerna.
Raden \emph{keepRunner: true} gör att filen \textunderscore SpecRunner.html sparas efter att
testerna körts. Filen kan sedan öppnas i en webbläsare och innehåller
detaljerad information om utfallet av testerna.
Med \emph{specs: 'test/*-spec.js'} valdes de testfall som skulle köras.
Alla filer som slutar med -spec.js i mappen test anses alltså vara
testfall som ska köras.
Raden \emph{template: 'test/template/spec-template.tmpl'} gör att testerna
körs med en speciell SpecRunner som även kan innehålla HTML-kod som testfallen
kan modifiera.
Eftersom Travis CI använder npm för att starta 
testerna så definierades testskriptet för npm med raden
\emph{''test'': ''grunt jasmine --verbose''} i filen package.json 
i projektets repository.

Testfallen skrevs med Jasmine. Jasmine har en enkel och intuitiv syntax.
Ett exempel på ett testfall skrivet med Jasmine följer nedan.

\begin{lstlisting}[
  basicstyle = \small
]
describe('The function splitOnce', function() {
	
  it('can split a string with a char correctly', function() {
    var str = 'a.b.c.d';
    var res = splitOnce(str, '.');
    expect(res[0]).toBe('a');
    expect(res[1]).toBe('b.c.d');
  });
}
\end{lstlisting}

På den första raden beskrivs vilken del av koden det är som ska testas.
På nästa rad beskrivs vad det är som ska testas i den utvalda delen av koden.
Med funktionen expect kontrolleras att koden har utfört testet på det sätt
som förväntats. Flera expect-funktioner kan användas i samma testfall.

Ett speciellt testfall skrevs
för att besvara den andra frågeställningen om hur effektivt det är 
att använda en webbaserad tjänst
för automatiserade tester. Det speciella testfallet visas nedan.

\begin{lstlisting}[
  basicstyle = \small
]
describe('Travis CI', function() {
	
  it('can do a lot of tests', function() {
    var date = new Date();
    var startTime = date.getTime();
    var time = startTime;
    var i = 0;

    while (time < startTime + 1000) {  
      var str = 'a.b.c.d';
      var res = splitOnce(str, '.');
      expect(res[0]).toBe('a');
      expect(res[1]).toBe('b.c.d');

      i++;
      date = new Date(); 
      time = date.getTime();
    };

    console.log(i);
  });
});
\end{lstlisting}

Testfallet testar funktionen splitOnce så många gånger som möjligt
under en sekund. Antalet gånger som funktionen hann köras skrivs
sedan ut på skärmen med en console.log. 
Testfallet har körts flera gånger på olika datum och olika tidpunkter.
Resultatet av testerna presenteras i \emph{Tabell E.5.1}
under rubriken \emph{E.5 Resultat}.
För att antalet ska
få en konkret betydelse visas även funktionen splitOnce nedan.

\begin{lstlisting}[
  basicstyle = \small
]
var splitOnce = function(str, split) {
  var index = str.indexOf(split);
  if (index === -1) {
    return [str, ''];
  }

  return [str.slice(0,index), str.slice(index + 1)];
};
\end{lstlisting}

Funktionen tar två paramterar. En sträng (str) som ska delas upp och 
en sträng (split) som anger vilket tecken eller vilken teckenkombination
som ska dela upp strängen. Funktionen delar endast upp strängen i två delar
även om (split) förekommer på flera positioner i (str). Funktionen returnerar 
en array med två element.
De två elementen är de två delarna av den ursprungliga strängen. Om den andra
parametern (split) inte existerar i den första parametern (str) så returneras
hela strängen i det första elementet och en tom sträng i det andra elementet.

För att besvara den tredje frågeställningen om vilka typer av tester som är 
svåra att utföra med en webbaserad tjänst så gicks koden igenom och 
försök till att skriva testfall genomfördes
med de olika delarna av koden. Resultatet av dessa tester redovisas under 
avsnittet \emph{E.5 Resultat}.

\subsection{Resultat}
Här presenteras resultatet av rapporten.

Resultatet av testerna som utfördes för att besvara den andra frågeställningen
visas i \emph{Tabell E.5.1}.

\noindent\emph{Tabell E.5.1}
\begin{center}
  \begin{tabular}{| l | l | l | l |}
  \hline
  Testnummer & Datum (ÅÅ-MM-DD) & Tid (hh-mm-ss) & Tester per sekund\\ \hline
  1 & 15-05-01 & 13:53:34 & 11380\\ \hline
  2 & 15-05-01 & 14:27:27 & 12462\\ \hline
  3 & 15-05-01 & 14:47:14 & 9386\\ \hline 
  4 & 15-05-03 & 10:21:57 & 14093\\ \hline 
  5 & 15-05-03 & 15:43:20 & 9875\\ \hline 
  6 & 15-05-04 & 09:42:39 & 10156\\ \hline 
  7 & 15-05-04 & 11:14:43 & 11933\\ \hline 
  8 & 15-05-04 & 17:42:34 & 10056\\ \hline 
  9 & 15-05-05 & 08:32:15 & 12216\\ \hline 
  10 & 15-05-05 & 08:55:02 & 9767\\ \hline 
  11 & 15-05-05 & 09:15:16 & 10153\\ \hline 
  12 & 15-05-05 & 10:12:57 & 6992\\ \hline 
  13 & 15-05-06 & 14:35:46 & 8703\\ \hline 
  14 & 15-05-06 & 17:01:37 & 10984\\ \hline 
  15 & 15-05-06 & 18:37:00 & 10186\\ \hline 
  \end{tabular}
\end{center}

Medelvärdet avrundat till närmsta heltal är: 10556 tester per sekund.

Standardavvikelsen avrundat till närmsta heltal är: 1710.

Den tid det tar från det att koden läggs upp på GitHub till dess att 
testkoden börjar köras är i genomsnitt ca en minut. Sedan tar det även ca 30
sekunder för alla node-moduler att installeras innan själva
testfallen börjar köras.

Undersökningen om vad om är svårt att testa ledde till följande resultat.
Det visade sig att vanliga Javascriptfunktioner är lätta
att testa så länge de ligger utanför jQueryfunktionen
\emph{\$(document).ready()}. Det kan illustreras med några
rader kod.

\begin{lstlisting}[
  basicstyle = \small
]
function easyToTest() {
  return 0;
}

$(document).ready(function() {

  $('#id1').click(
    var test = easyToTest(); 
  );

  $('#id2').click(
    //Hard to test
    var test = 0; 
  );
});

\end{lstlisting}

Det är alltså rekommenderat att skriva alla Javascriptfunktioner utanför 
jQueryfunktionen \emph{\$(document).ready()} eftersom koden då blir 
lättare att testa.

JQueryfunktioner är i allmänhet svårare att testa än vanliga
Javascriptfunktioner. Anledningen är att jQueryfunktioner
startas av och manipulerar HTML-objekt. Ett sätt att lösa detta är
att använda en egen \textunderscore Specrunner.html. Då kan HTML-objekten i den
filen användas för att testa jQueryfunktionerna.

\subsection{Diskussion}
Under den här rubriken diskuteras rapportens resultat
och metod.

\subsubsection{Resultat}
Resultatet angående hur effektivt det var att använda Travis CI
var inte särskilt förvånande. Att den lilla testfunktionen kunde köras
ca $10^4$ gånger per sekund verkar rimligt. Det intressanta med resultatet
var att det kan dröja ca en minut från att koden läggs upp på GitHub
till dess att testfallen börjar köras och att effektiviteten
kan variera med ca $\pm 30 \%$ vid olika tidpunkter.

Det är även värt att påpeka att hur koden skrivs och struktureras
i väldigt hög grad påverkar hur lätt den blir att testa. I alla fall
med den metod som använts under arbetet med den här rapporten. Det är 
därför en god idé att tidigt i ett projekt sätta upp riktlinjer för hur
koden ska skrivas på ett sätt som gör den lätt att testa. Då kan man
tjäna in tid som kanske annars hade behövt användas till att 
skriva om koden senare i projektet.

\subsubsection{Metod}
Den metod för testning som användes under arbetet med den här rapporten var
medvetet väldigt enkel och inte särskilt kraftfull. Anledningen var
att det var mycket annat som skulle göras i projektet och tiden var
begränsad till 300 timmar. I efterhand visade det sig dock att det hade varit 
möjligt att lägga lite mer tid på att använda en lite
mer avancerad metod för att kunna testa en större
del av koden i projektet. Nu användes den överblivna tiden istället till
att göra fler manuella tester av koden.

Det skulle inte vara några problem att skala metoden till ett större projekt.
Men i ett större projekt skulle det antagligen finnas någon som arbetar
uteslutande med testning och då skulle det finnas tid till att använda en
mer avancerad och mer kraftfull metod. Därför är det en bättre idé att 
använda den här metoden i ett mindre projekt.

Det är även värt att nämna ett en nackdel med metoden som använts för tester
under projektet
är att om ett eventuellt fel upptäcks så är koden redan upplagd på GitHub.
Det hade varit bättre om koden kunde testas innan den lades upp på GitHub
eftersom det då skulle gå att förebygga fel istället för att upptäcka 
dem i efterhand.

\subsection{Slutsatser}
Under den här rubriken presenteras svaren på
frågeställningarna och framtida arbete som skulle
kunna utföras inom det område som behandlats i rapporten.

\subsubsection{Hur kan man använda webbaserade tjänster för
att utföra kontinuerliga automatiserade tester av webbapplikationer?}
Ett sätt att använda en webbaserad tjänst för att utföra kontinuerliga 
automatiserade tester av en webbapplikation är att använda  
Travis CI tillsammans med Jasmine på det sätt som beskrivits
under avsnittet \emph{E.4 Metod}. Observera att den webbapplikation som testades
var skriven med Javascript. Node.js användes på serversidan och 
jQuery användes på klientsidan.

\subsubsection{Hur effektivt är det att använda en webbaserad tjänst
för automatiserade tester?}
Det enkla testfallet som användes kunde köras ca $10^4$ gånger per sekund och
det tar ca 60 sekunder från att koden läggs upp på GitHub till det att den
börjar köras på servern.

Om man använder ett antal testfall (med liknande komplexitet som
det som användes i den här rapporten) i storleksordningen $60 \cdot 10^4$ eller
mindre så kommer den största väntetiden vara tiden från att koden
läggs upp på GitHub till det att den börjar köras på servern. Själva
testfallen kommer att köras på mindre än en minut.

\subsubsection{Vilka typer av tester är svåra att utföra 
med en sådan tjänst?}
Kortfattat kan man säga att vanliga Javascriptfunktioner är
väldigt enkla att testa med de metoder, ramverk och tjänster som
använts under arbetet med den här rapporten. Men jQueryfunktioner och
Javascriptfunktioner i jQueryfunktioner är svårare att
testa. För en mer detaljerad beskrivning se rubriken \emph{E.5 Resultat}.

\subsubsection{Framtida arbete inom området}
Det finns mycket arbete som skulle kunna utföras inom området som behandlats
i den här rapporten.

Det vore intressant att undersöka hur stor andel av de totala felen
som upptäcks med den typ av testning som har beskrivits i den här rapporten.
Det kräver dock mer tid och resurser än vad som var tillgängligt under
arbetet den här rapporten.

Det vore även intressant att undersöka vilka nackdelar och fördelar det
finns med webbaserade testmiljöer jämfört med lokala
testmiljöer.

En annan undersökning som skulla kunna utföras är att jämföra
flera olika ramverk för testning av Javascript. Till exempel skulle
man kunna jämföra Jasmine och Mocha. Man skulle även kunna jämföra
flera olika webbaserade tjänster för automatiserade tester. Till
exempel skulle man kunna jämföra Travis CI med Codeship och Drone.io.

%\subsection{Referenser}
%\vspace{-9mm}
%\renewcommand{\refname}{}
%\begin{thebibliography}{9}

%\bibitem{Royce}
%W.W. Royce, ''Managing the development of large software systems,''
%\textit{Proceedings of IEEE WESCON}, pp. 2, aug, 1970.
%[Online].
%Tillgänglig (nytryckt med annan sidnumrering):
%\url{http://www.cs.umd.edu/class/spring2003/cmsc838p/Process/waterfall.pdf}.
%[Hämtad april 28, 2015].

%\bibitem{Karlsson}
%O. Karlsson, ''Automatiserad testning av webbapplikationer,''
%Linköpings univ., Linköping, Sverige, 2014, pp. 43.
%[Online]. 
%Tillgänglig: 
%\url{http://www.diva-portal.org/smash/get/diva2:727654/FULLTEXT01.pdf}.
%[Hämtad april 19, 2015].

%\end{thebibliography}

\newpage
\section{Checkning av checklistor - Robin Andersson}
\subsection{Inledning}
Vårt system ska innehålla olika typer av checklistor på olika webbsidor. Om flera användare är inne på samma sida samtidigt och en person checkar en checkruta så ska den checkrutan bli checkad för alla användare som är inne på den webbsidan.\\

Checklistan kommer att implementeras med hjälp av html, javascript och jquery samt Socket.IO.

\subsubsection{Syfte}
Syftet med denna del av projektet är att flera sjuksköterskor samtidigt ska kunna förbereda operationer genom att plocka olika artiklar samt förbereda operationssalen och checka av det som är utfört utan att det ska bli några konflikter med att flera sjuksköterskor plockar samma artikel eller liknande.

\subsubsection{Frågeställning}
\begin{itemize}
\item Går det att anpassa checklistan för en surfplatta medan den samtidigt innehåller information om var artiklar befinner sig samt hur många av varje artikel som behövs?
\item Kommer Socket.IO vara tillräckligt snabbt för att flera personer ska kunna checka av artiklar samtidigt utan förvirring?
\end{itemize}

\subsubsection{Avgränsningar}
Eftersom denna del av projektet endast innehåller checkande av checklistor så saknas etiska aspekter.

\subsection{Teori}
Huvuddelen i implementeringen av checklistan är kommunikationen med Socket.IO som använder sig av websockets för att kommunicera mellan front-end och back-end. Information om Socket.IO finns på webbplatsen: \textit{http://socket.io/}

 
\subsection{Metod}
Jag började med att fundera på hur kommunikationen skulle fungera på för sätt. Jag skissade ner olika förslag på ett papper och kom på det sättet fram till hur jag skulle implementera kommunikationen. Sedan implementerade jag den och fick den att fungera. Därefter så refaktoriserade jag koden för att få den kortare och mer lättläst.

\subsection{Resultat}
Jag kom fram till att när en användare går in på en operationsförberedelse så kommer denne in i ett rum. Varje gång en person sedan checkar en checkbox så skickas ett Socket.IO meddelande till servern som innehåller information om vilken checkruta som ska checkas samt vilket rum checkboxen ska checkas i. Servern skickar sedan ett meddelande till det givna rummet vilken checkruta som ska checkas och alla klienter som är anslutna till det rummet checkar den givna checkrutan. \\

Figuren nedan visar detta flöde i ett sekvens liknande box-and-line-diagram.
\includegraphics[scale=0.5]{checklistdiagram}\\

När två klienter går in på en operation och en klient checkar en checkruta för första gången tar det strax under en sekund innan checkrutan checkas för den andra klienten. Därefter när någon klient checkar en checkruta så kan jag inte se någon fördröjning alls från det att en klient checkar en checkruta och en annan klient får den checkrutan checkad.\\

Kunden har prövat att ha flera personer inne på samma plocklista samtidigt och checka av olika artiklar. Kunden tyckte att det fungerade bra och påpekade inte någon fördröjning. \\

All information som krävs för plocklistorna fick plats utan att det blev för plottrigt.

\subsection{Diskussion}
\subsubsection{Resultat}
Eftersom jag endast skickar data om vilken checkruta som ska checkas till de klienter som är inne på den operation som checkrutan blev checkad på så uppdateras checkningar snabbare än att göra den enkla lösningen att bara skicka datat till alla anslutna klienter. Att det tar nästan en sekund för en checkning att uppdateras på andra klienter för första gången är långsammare än förväntat. Men eftersom det endast gäller just första artikeln och att kunden har testat checkning med flera personer samtidig utan att märka några problem så verkar detta inte vara något praktiskt problem. Att en checkning sedan kan uppdateras nästan helt utan fördrdröjning var bättre än vad jag hade förväntat mig.

\subsubsection{Metod}
Den metod jag använde mig av fungerade bra, men jag tror att jag skulle kunnat komma fram till samma resultat snabbare genom att göra kortare funktioner och vettigare namn redan från början istället för att göra något som funkar så snabbt som möjligt och sedan refaktorisera. För nu blev det väldigt förvirrande kod från början och jag var tvungen att sitta och tänka på vad kod jag skrivit faktiskt gjorde. Men att skissa olika förslag på ett papper först tror jag var en väldigt bra idé, det gjorde att jag fick några möjliga lösningar och sedan kunde jag överväga fördelar och nackdelar med de olika lösningarna för att sedan välja den som verkade bäst.

\subsection{Slutsatser}
Eftersom det fungerar bra med checkning av checklistor med hjälp av socket.io och all nödvändig information får plats utan att det upplevs som plottrigt så uppfylls syftet med denna del av projektet och min frågeställning har blivit besvarad.

\subsection{Referenser}
\vspace{-9mm}
\begin{thebibliography}{9}

\end{thebibliography}
\newpage
\section{Titel}
\subsection{Inledning}

\subsubsection{Syfte}
Syftet med denna enskilda utredningen är att undersöka olika kvalitetsaspekter då ett mindre projekt utförs av en liten grupp utvecklare.
\subsubsection{Frågeställning}
\begin{itemize}
\item Är kvalitet viktigt i små projekt utförda av små projektgrupper?
\item Vilka kvalitetaspekter kan man kompromissa bort i mindre projekt?
  
\end{itemize}
\subsubsection{Avgränsningar}
\subsection{Bakgrund}
\subsection{Teori}
\subsection{Metod}
\subsection{Resultat}
\subsection{Diskussion}
\subsubsection{Resultat}
\subsubsection{Metod}
\subsection{Slutsatser}

\newpage

\part{Referenser}
\vspace{-15mm}
\renewcommand{\refname}{}
\begin{thebibliography}{9}

%GEMENSAMMA REFERENSER

\bibitem{apache}
Apache 2.0 [Online] Tillgänlig \url{https://github.com/jcmandersson/Operation/blob/master/LICENSE} [Hämtad 2015-05-13]

\bibitem{ieee830}
IEEE Std 830-1998: IEEE Recommended Practice for Software Requirements Specifications [Online] Tillgänglig:
\url{http://www.math.uaa.alaska.edu/~afkjm/cs401/IEEE830.pdf} [Hämtad 2015-05-13]

\bibitem{socketio}
Socket.IO [Online] Tillgänglig: 
\url{http://socket.io/} [Hämtad 2015-05-12]

\bibitem{bootstrap}
Bootstrap [Online] Tillgänglig: 
\url{http://getbootstrap.com/} [Hämtad 2015-05-12]

\bibitem{jquery}
jQuery [Online] Tillgänglig: 
\url{https://jquery.com/} [Hämtad 2015-05-12]

\bibitem{less}
Less [Online] Tillgänglig: 
\url{http://lesscss.org/} [Hämtad 2015-05-12]

\bibitem{handlebars}
Handlebars [Online] Tillgänglig: 
\url{http://handlebarsjs.com/} [Hämtad 2015-05-12]

\bibitem{angular}
AngularJS [Online] Tillgänglig: 
\url{https://angularjs.org/} [Hämtad 2015-05-12]

%RAPPS REFERENSER
\bibitem{mongoPerf} Parker, Zachary and Poe, Scott and Vrbsky, Susan V., ''Comparing NoSQL MongoDB to an SQL DB'' vid \textit{Proceedings of the 51st ACM Southeast Conference}, New York, ACM, 2013, 5:1--5:6. Tillänglig: \url{http://dl.acm.org/citation.cfm?id=2500047} [Hämtad 2015-05-11].

%JONAS REFERENSER
\bibitem{php_history}
History of PHP [Online] Tillgänglig:
\url{http://php.net/manual/en/history.php.php}
[Hämtad 2015-05-11]
	
\bibitem{python_flask}
Flask [Online] Tillgänglig:
\url{http://flask.pocoo.org/}
[Hämtad 2015-05-11]
	
\bibitem{node_performance}
K. Fysarakis et al, ''Node.DPWS: High performance and scalable Web Services for the IoT'', Crete, Greece, 2015


%PÅLS REFERENSER
\bibitem{m_spichkova}
M. Spichkova et al, “Do we really need to write documentation for a system?” presented at the In Proceedings International Conference on Model-Driven Engineering and Software Development., Barcelona, 2013.

\bibitem{docs_launch}
Google Announces limited test on Google Labs [Online] Tillgänglig: 
\url{http://googlepress.blogspot.se/2006/06/google-announces-limited-test-on-google_06.html} [Hämtad 2015-04-28]

\bibitem{upstartle}
Google purchases Upstartle [Online] Tillgänglig: 
\url{http://googleblog.blogspot.se/2006/03/writely-so.html} [Hämtad 2015-04-28]

\bibitem{donald_knuth}
Tex origin [Online] Tillgänglig: 
\url{https://gcc.gnu.org/ml/java/1999-q2/msg00419.html} [Hämtad 2015-04-20]

\bibitem{leslie_lamport}
LaTeX: A Document Preparation System [Online] Tillgänglig: 
\url{http://research.microsoft.com/en-us/um/people/lamport/pubs/pubs.html#latex} [Hämtad 2015-04-20]

\bibitem{overleaf}
Overleaf 
\url{https://www.overleaf.com/} [Hämtad 2015-04-20]

\bibitem{authorea}
Authorea 
\url{https://www.authorea.com/} [Hämtad 2015-04-28]
 
\bibitem{latex_standard}
LaTeX: Standard for typesetting scientific documents
\url{ftp://ftp.dante.de/tex-archive/info/intro-scientific/scidoc.pdf} Sida 2. [Hämtad 2015-04-28]


%FALKS REFERENSER
\bibitem{Hull}
Hull. E, Ken. J och Jeremy. D, Requirements Engineering, Third edition. London: Springer, 2011.

\bibitem{Arvola}
Arvola, M. (2014) Interaktionsdesign och UX: Om att skapa en god användarupplevelse. Studentlitteratur AB, Lund.

\bibitem{Dorfman}
Dorfman, M. (1997) Requirements Engineering. Institute of Electrical and Electronics Engineers, Inc. 

\bibitem{Lauesen}
Lauesen, S. (2002) Software Requirements: Styles and Techniques. Harlow: AddisonWessly.


%ERIKS REFERENSER
\bibitem{Royce}
W.W. Royce, ''Managing the development of large software systems,''
\textit{Proceedings of IEEE WESCON}, pp. 2, aug, 1970.
[Online].
Tillgänglig (nytryckt med annan sidnumrering):
\url{http://www.cs.umd.edu/class/spring2003/cmsc838p/Process/waterfall.pdf}.
[Hämtad april 28, 2015].

\bibitem{Karlsson}
O. Karlsson, ''Automatiserad testning av webbapplikationer,''
Linköpings univ., Linköping, Sverige, 2014, pp. 43.
[Online]. 
Tillgänglig: 
\url{http://www.diva-portal.org/smash/get/diva2:727654/FULLTEXT01.pdf}.
[Hämtad april 19, 2015].

%ROBINS REFERENSER
\bibitem{websocketbook} Wang Vanessa, Salim Frank, Moskovits Peter; The Definitive Guide to HTML5 WebSocket; New York City APress, 2013.
\bibitem{websocketreport} http://www.scirp.org/journal/PaperInformation.aspx?PaperID=25428

%ALBERTS REFERENSER

\bibitem{expressRouter}
Express 4.x api [Online] Tillgänglig: 
\url{http://expressjs.com/4x/api.html} [Hämtad 2015-05-11]

\bibitem{sqlMS}
Microsoft Driver for Node.js for SQL Server Preview [Online] Tillgänglig: 
\url{https://www.microsoft.com/en-us/download/details.aspx?id=29995} [Hämtad 2015-05-11]

\bibitem{mongoComp}
Who uses MongoDB? [Online] Tillgänglig: 
\url{http://www.mongodb.com/who-uses-mongodb} [Hämtad 2015-05-11]

\bibitem{databaseStats}
DB-engines Ranking [Online] Tillgänglig: 
\url{http://db-engines.com/en/ranking} [Hämtad 2015-05-11]

\bibitem{compareSQL} Nance Cory, Losser Travis, Iype Reenu, Harmon Gary; NoSQL vs RDBMS - Why There Is Room For Both; SAIS 2013 Proceedings; Tillgänglig: 
\url{http://aisel.aisnet.org/cgi/viewcontent.cgi?article=1026&context=sais2013} [Hämtad 2015-05-13]

\bibitem{sqlGit}
Github för node-sqlserver [Online] Tillgänglig:  
\url{https://github.com/Azure/node-sqlserver} [Hämtad 2015-05-11]

\bibitem{sqlBugs}
Github issues för node-sqlserver [Online] Tillgänglig: 
\url{https://github.com/Azure/node-sqlserver/issues} [Hämtad 2015-05-11]

\end{thebibliography}

\newpage

\part{Appendix A - Användarutvärdering}
\includepdf[pages={-}]{utv.pdf}
\end{document}
